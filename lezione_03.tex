\section{Sistemi elettrici}
Si richiamano le convenzioni utilizzate nei sistemi elettrici, saranno composti da uno o più
generatori di tensione o corrente e composti da bipoli passivi.

Verrà utilizzata la convenzione dell'utilizzatore per i seguenti bipoli
\begin{figure}[h]
\centering
\begin{circuitikz}
\draw (0,2) [R=$R$,i>^=$i$,v<=$v$] to (0,0);
\draw (1.5,1) node[label]{$v = Ri$};
\draw (4,2) [C=$C$,i>^=$i$,v<=$v$] to (4,0);
\draw (5.8,1) node[label]{$q = Cv$};
\draw (7.5,2) [L=$L$,i>^=$i$,v<=$v$] to (7.5,0);
\draw (9,1) node[label]{$\Phi = Li$};
\end{circuitikz}
\end{figure}

Derivando le equazioni dei bipoli dinamici rispetto al tempo si ottiene
$$\begin{aligned}
q &= Cv\stackrel{\frac{d}{dt}}{\rightarrow} \frac{dq}{dt} = i =
\frac{d}{dt} \left(Cv\right) \stackrel{C\text{ cost}}{=} C\frac{dv}{dt} = C\dot{v}\\
\Phi &= Li \stackrel{\frac{d}{dt}}{\rightarrow} \frac{d\Phi}{dt} = v = \frac{d}{dt}\left(Li \right)
\stackrel{L\text{ cost}}{=} L \frac{di}{dt} = L \dot{i}
\end{aligned}$$

Si aggiungono ai bipoli passivi i generatori di tensione e corrente
\begin{figure}[h]
\centering
\begin{circuitikz}[american voltages]
\draw (0,0) [european voltage source] to (0,2);
\draw (0.5,2) to [open, v^>=$ $,l=$e$] (0.5,0);
\draw (4,2) [current source,i<^=$i$] to (4,0);
\end{circuitikz}
\end{figure}

\subsection{Esempio rete RLC}
Si ha la seguente rete
\begin{figure}[h]
\centering
\begin{circuitikz}
\draw (0,0) to [voltage source,v=$e$]  (0,2)
            to [L=$L$,i>^=$i_L$,-*] (2,2)  node[above,label]{$A$}
            to [C,l_=$C$,i>_=$i_C$] (2,0)
            to (0,0);
\draw (2,2) to (3,2) to [R=$R$,i>^=$i_R$] (3,0) to (2,0);
\end{circuitikz}
\end{figure}

Si ricavano le equazioni del sistema utilizzando in questo caso le leggi di Kirchhoff applicate al
nodo $A$ e alla prima maglia (che comprende il generatore)
$$\left\{\begin{aligned}
i_L &= i_C + i_R &\Rightarrow&\ i_L = C\dot{v}_C + \frac{v_C}{R}\\
e &= v_L + v_C &\Rightarrow&\ e = L\dot{i}_L + v_C\\
v_C &= v_R
\end{aligned}\right.
$$

Elaborando le equazioni si vede che alcune variabili potranno risultare superflue, ad esempio in
questo caso $v_C$ e $v_R$ sono la stessa grandezza.

Si individuano ora le variabili e si dividono in ingressi e uscite al fine di rispettare il
principio di causalità ricordando che gli ingressi compaiono differenziati un numero minore di volte
$$\begin{matrix}
e& i_L& v_C\\
(0) & (1) & (1)\\
u & y & y\\
u & x_1 & x_2
\end{matrix}
$$

Si scelgono le variabili di stato, applicando la legge di Cauchy, il numero di ingressi è pari a
due perché entrambe le variabili compaiono differenziate una sola volta.
$$
n=2,\ x_1 = i_L,\ x_2 = v_C
$$

Si scrive quindi la ISU (il sistema è tempo invariante quindi le funzioni della ISU non dipendono
da $t$
$$\begin{aligned}
\dot{x}_1 &= f_1(x_1,x_2,u) = \dot{i}_L = \frac{e}{L} - \frac{v_C}{L} = -\frac{1}{L}x_2 +
\frac{1}{L} u\\
\dot{x}_2 &= f_2(x_1,x_2,u) = \dot{v}_C = \frac{i_L}{C} - \frac{v_C}{RC} = \frac{1}{C}x_1 -
\frac{1}{RC}x_2
\end{aligned}$$

Andrebbero aggiunte le relazioni che legano le uscite all'effettiva variabile di studio ad esempio
la tensione ai capi della resistenza e la corrente nell'induttanza, dunque
$$
y_1 = v_R,\ y_2 = i_L
$$
ma riferendosi al modello ISU
$$\begin{aligned}
y_1 &= g_1(x_1,x_2,\cancel{u}) = v_R = v_C = x_2\\
y_2 &= g_2(x_1,x_2,\cancel{u}) = i_L = x_1
\end{aligned}$$

Si vuole ora classificare la ISU, si verifica la linearità, tutte le funzioni $f$ e $g$ sono
lineari nello stato e negli ingressi, di conseguenza il sistema è lineare, è anche tempo invariante
dato che i parametri del sistema sono costanti nel tempo.

Si afferma ora che il sistema è strettamente proprio, ossia nell'espressione della $g$
non compare l'ingresso $u$.

Si riscrive il sistema in forma compatta utilizzando la notazione
\ref{eq.:ISU_compatta_matrice} ponendo
$$\begin{aligned}
x &= \begin{pmatrix}x_1\\x_2 \end{pmatrix},& u &= \begin{pmatrix} u \end{pmatrix},&
y&=\begin{pmatrix}
y_1 \\y_2 \end{pmatrix}\\
n&=2& m&=1&  p&=2
\end{aligned}
$$
dunque
\begin{equation}\left\{\begin{aligned}
&\dot{x} = \begin{bmatrix}
0 & -\frac{1}{L}\\
\frac{1}{C} & -\frac{1}{RC}
\end{bmatrix} x + \begin{bmatrix}\frac{1}{L}\\0 \end{bmatrix} u\\
&y =\begin{bmatrix}
0 & 1\\
1 & 0
\end{bmatrix} x + \begin{bmatrix}0\\0 \end{bmatrix}u
\end{aligned}\right.\end{equation}
\newpage

\subsection{Esempio rete RLC 2}
Si analizza il seguente circuito
\begin{figure}[h]
\centering
\begin{circuitikz}
\draw (0,0) to [current source,i=$i_1$] (0,2) to (2,2) node[above,label]{$A$}
            to [R=$R_1$,i>_=$i_{R_1}$,v<=$v_1$,*-] (2,0) to (0,0);
\draw (2,2) to (4,2) to [C=$C_1$,i>_=$i_{C_1}$] (4,0) to (2,0);
\draw (4,2) to [L=$L$,i>^=$i_L$] (6,2)
            to [C=$C_2$,i>^=$i_{C_2}$,v<=$v_{C_2}$] (6,0) to (4,0);
\draw (6,2) to (8,2) node[above,label]{$B$} to [R=$R_2$,i>^=$i_{R_2}$,*-] (8,0) to (6,0);
\draw (8,2) to (10,2) to [current source,i>^=$i_2$] (10,0) to (8,0);
\end{circuitikz}
\end{figure}

Applicando le leggi di Kirchhoff
$$\begin{aligned}
A: i_1 &= i_{R_1} + i_{C_1} + i_L\\
B: i_L &= i_{C_2} + i_{R_2} + i_2 \\
M: v_1 &= v_L + v_2
\end{aligned}
$$

Sono presenti più variabili che equazioni, si utilizzano le equazioni caratteristiche dei bipoli per ridurre il sistema, sviluppando le precedenti si ottiene:
$$
\begin{aligned}
 A: &= \frac{v_1}{R_1} + C_1\dot{v}_1 + i_L \\
 B: &= C_2\dot{v}_2 + \frac{v_2}{R_2} + i_2 \\
 M: &= v_1 = v_2 + v_2 = L\dot{i_L} + v_2
\end{aligned}
$$

Ora si indicano le variabili presenti nelle espressioni appena ricavate e vanno classificate tra causa ed effetto, contando il massimo ordine di differenziazione di ciascuna:
$$\begin{aligned}
 i_1& &i_2& & v_1& &v_2& &i_L \\
 (0)& &(0)& & (1)& &(1)  & &(1)
\end{aligned}$$
Le prime due variabili saranno le cause, le restanti le uscite o gli effetti.

L'ordine del sistema $(n)$ è pari alla somma degli ordini di differenziazione delle variabili non di ingresso, ossia $3\times 1 = 3$

1:04:59
