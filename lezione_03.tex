\section{Sistemi elettrici}
Si richiamano le convenzioni utilizzate nei sistemi elettrici, saranno composti da uno o più
generatori di tensione o corrente e composti da bipoli passivi.

Verrà utilizzata la convenzione dell'utilizzatore per i seguenti bipoli
\begin{figure}[h]
\centering
\begin{circuitikz}
\draw (0,2) [R=$R$,i>^=$i$,v<=$v$] to (0,0);
\draw (1.5,1) node[label]{$v = Ri$};
\draw (4,2) [C=$C$,i>^=$i$,v<=$v$] to (4,0);
\draw (5.8,1) node[label]{$q = Cv$};
\draw (7.5,2) [L=$L$,i>^=$i$,v<=$v$] to (7.5,0);
\draw (9,1) node[label]{$\Phi = Li$};
\end{circuitikz}
\end{figure}

Derivando le equazioni dei bipoli dinamici rispetto al tempo si ottiene
$$\begin{aligned}
q &= Cv\stackrel{\frac{d}{dt}}{\rightarrow} \frac{dq}{dt} = i =
\frac{d}{dt} \left(Cv\right) \stackrel{C\text{ cost}}{=} C\frac{dv}{dt} = C\dot{v}\\
\Phi &= Li \stackrel{\frac{d}{dt}}{\rightarrow} \frac{d\Phi}{dt} = v = \frac{d}{dt}\left(Li \right)
\stackrel{L\text{ cost}}{=} L \frac{di}{dt} = L \dot{i}
\end{aligned}$$

Si aggiungono ai bipoli passivi i generatori di tensione e corrente
\begin{figure}[h]
\centering
\begin{circuitikz}[american voltages]
\draw (0,0) [european voltage source] to (0,2);
\draw (0.5,2) to [open, v^>=$ $,l=$e$] (0.5,0);
\draw (4,2) [current source,i<^=$i$] to (4,0);
\end{circuitikz}
\end{figure}
