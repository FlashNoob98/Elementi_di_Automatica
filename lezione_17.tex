
\subsection{Analisi delle risposte dei sistemi del secondo ordine}
Si considera il caso di un sistema del secondo ordine con due poli reali e
distinti, ha la seguente funzione di trasferimento
$$
W(s) =\frac{K_B}{(1+s\tau_1)(1+s\tau_2)}
$$
con $\tau_1 > \tau_2$,
\begin{figure}[h]
 \centering
 \begin{tikzpicture}
  \begin{axis}[
   width=0.6\linewidth,
   height=0.3\linewidth,
   axis lines = middle,
   xtick={-2,-1,0},
   ytick={0},
   xlabel={\Re},
   ylabel={\Im},
   xmax=0.3,xmin=-2.5,ymin=-0.3,ymax=1,
   xticklabels={$p_2$,$p_1$,0},
   ]
\filldraw(-1,0) circle (2pt)
         (-2,0) circle (2pt);
  \end{axis}
 \end{tikzpicture}
\end{figure}
dunque $p_1$ è il polo dominante.

Si analizza la risposta all'impulso
$$
W(t) = \Lap^{-1}[W(s)] =
\frac{K_B}{\tau_1\tau_2}\Lap^{-1}\left[\frac{R_1}{s+\frac{1}{\tau_1}} +
\frac{R_2}{s+\frac{1}{\tau_2}}\right]
$$
L'eccesso poli zeri è pari a due, dunque la sommatoria dei residui deve essere
pari a zero. $R_1 = \frac{\tau_1\tau_2}{\tau_1-\tau_2} \rightarrow R_2 = -R_1 =
\frac{\tau_1\tau_2}{\tau_2-\tau_1}$.
$$
W(t) =
K_B\frac{1}{\tau_1-\tau_2}\left(e^{-\frac{t}{\tau_1}}-e^{-\frac{t}{\tau_2}}
\right)\delta_{-1}(t)
$$

05:29
