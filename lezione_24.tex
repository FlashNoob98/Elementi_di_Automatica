
\section{Raggiungibilità ed osservabilità}
Queste due proprietà cercano di studiare in modo qualitativo le potenzialità
dell'ingresso di un certo sistema ISU e la possibilità che offre l'uscita
rispetto alla possibilità di capire cosa accade nel sistema osservando l'uscita.
Le potenzialità dell'ingresso possono influenzare l'evoluzione dello stato,
racchiuso in un sottospazio $X^n$ dell'intero sistema, non è infatti detto che
l'ingresso possa modificare senza vincoli tutte le variabili di stato, questo
fenomeno viene studiato attraverso la proprietà di raggiungibilità.

Si vuole inoltre capire se è possibile ricostruire l'evoluzione interna dello
stato attraverso l'osservazione dell'uscita, in generale non è possibile, si può
ricostruire una parte dello stato e non tutto.

\subsection{Raggiungibilità (LTI)}
Uno stato $\hat{x}$ è raggiungibile se e solo se
$$
\hat{x} \text{ raggiungibile} \Leftrightarrow  \exists \hat{t},
\hat{u}_{[t_0,\hat{t}]} : \left.x_f(\hat{t})\right|_{u=\hat{u}} = \hat{x}
$$
esiste un tempo $\hat{t}$ ed un ingresso $\hat{u}$ tali per cui l'evoluzione
forzata $x_f(\hat{t})$ coincida con lo stato $\hat{x}$.
Questa caratteristica equivale a dire che è possibile pilotare il sistema
mediante l'ingresso in un certo intervallo finito e portarlo proprio nello
stato desiderato.
Un sistema in cui tutti gli stati sono raggiungibili si dirà
\textit{completamente raggiungibile}, si definisce il sottospazio di
raggiungibilità $X_R$ sottoinsieme dello spazio dello stato.
$$
X_R = \{ x\in X^n : x \text{ è raggiungibile}\} \subseteq X^n
$$
Se il sottospazio di raggiungibilità coincide con lo spazio dello stato allora
il sistema nella sua forma ISU è completamente raggiungibile.
La raggiungibilità è fondamentale per un progettista, permette di capire se si
hanno a disposizione i giusti ingressi per il sistema in esame e per
modificarne lo stato.

Si consideri la funzione della risposta forzata dalle formule di Lagrange
$$
x_f(t) = \int_{t_0}^t e^{A(t-t_0)}Bu(\tau) d\tau
$$
Nell'espressione compaiono solo le matrici $A$ e $B$, dunque la proprietà di
raggiungibilità dipende solo da queste due matrici. La matrice $A$ contiene i
modi naturali e le caratteristiche interne al sistema, la matrice $B$ invece
modula l'ingresso e lo lega alla variabile di stato.

