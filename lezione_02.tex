%lezione_02.tex
\chapter{Modelli di sistemi dinamici}
I principali modelli di sistemi dinamici sono due:
\begin{itemize}
\item [$(IU)$]Ingresso-uscita
\begin{equation}
f\left(y^{(n)},y^{(n-1)},\dots,y,u^{(m)},u^{(m-1)},\dots,u\right)=0
\label{eq.:modello_ingresso_uscita}
\end{equation}
con $m<n$ (per la regola di Cauchy) per avere un sistema strettamente causale. Con $m=n$ un sistema
è causale ma non strettamente causale, in genere tutti i modelli fisici sono strettamente causali.
La \ref{eq.:modello_ingresso_uscita} prende il nome di \textit{equazione generale del sistema}
mentre $n$
prende il nome di \textit{ordine del sistema}.
La funzione può anche essere un sistema di equazioni differenziali di ordine inferiore.
\item[$(ISU)$] Ingresso-stato-uscita
\begin{equation}
y(t) = x_1(t), \dot{y}(t) = x_2(t),\dots,y^{(n-1)}(t)=x_n(t)
\label{eq.:modello_ingresso_stato_uscita}
\end{equation}
Compaiono in questo modello tre tipologie di variabili, \textit{ingresso} e \textit{uscita} come nel
caso precedente e le variabili \textit{di stato} necessarie a completare il problema di Cauchy.
Un possibile insieme di variabili di stato sono quelle necessarie a specificare le condizioni
iniziali dell'equazione differenziale. Questo insieme può essere definito come \textit{stato} del
sistema.
Nella \ref{eq.:modello_ingresso_stato_uscita} si vede che il numero di variabili di stato $(x_n)$
necessarie alla risoluzione del problema è pari all'ordine del sistema.
\end{itemize}

Qualunque sistema dinamico può essere rappresentato con una diversa scelta delle variabili di
stato, varieranno le equazioni ma il modello resta valido per quel determinato sistema. La
rappresentazione ingresso-uscita è invece unica.

Si vede che nella rappresentazione $ISU$, una volta ottenuta l'equazione del sistema, questa può
essere trasformata per evidenziare particolari proprietà del sistema.

\newpage
\section{Costruzione di un modello ISU}
Si può costruire seguendo quattro passaggi
\begin{enumerate}
\item Scrivere tutte le equazioni del sistema (eq. diff.)
\item 18:45
\end{enumerate}
