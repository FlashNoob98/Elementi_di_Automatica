%lezione_02.tex
\chapter{Modelli di sistemi dinamici}
I principali modelli di sistemi dinamici sono due:
\begin{itemize}
\item [$(IU)$]Ingresso-uscita
\begin{equation}
f\left(y^{(n)},y^{(n-1)},\ldots,y,u^{(m)},u^{(m-1)},\ldots,u\right)=0
\label{eq.:modello_ingresso_uscita}
\end{equation}
con $m<n$ (per la regola di Cauchy) per avere un sistema strettamente causale. Con $m=n$ un sistema
è causale ma non strettamente causale, in genere tutti i modelli fisici sono strettamente causali.
La \ref{eq.:modello_ingresso_uscita} prende il nome di \textit{equazione generale del sistema}
mentre $n$
prende il nome di \textit{ordine del sistema}.
La funzione può anche essere un sistema di equazioni differenziali di ordine inferiore.
\item[$(ISU)$] Ingresso-stato-uscita
\begin{equation}
y(t) = x_1(t), \dot{y}(t) = x_2(t),\ldots,y^{(n-1)}(t)=x_n(t)
\label{eq.:modello_ingresso_stato_uscita}
\end{equation}
Compaiono in questo modello tre tipologie di variabili, \textit{ingresso} e \textit{uscita} come nel
caso precedente e le variabili \textit{di stato} necessarie a completare il problema di Cauchy.
Un possibile insieme di variabili di stato sono quelle necessarie a specificare le condizioni
iniziali dell'equazione differenziale. Questo insieme può essere definito come \textit{stato} del
sistema.
Nella \ref{eq.:modello_ingresso_stato_uscita} si vede che il numero di variabili di stato $(x_n)$
necessarie alla risoluzione del problema è pari all'ordine del sistema.
\end{itemize}

Qualunque sistema dinamico può essere rappresentato con una diversa scelta delle variabili di
stato, varieranno le equazioni ma il modello resta valido per quel determinato sistema. La
rappresentazione ingresso-uscita è invece unica.

Si vede che nella rappresentazione $ISU$, una volta ottenuta l'equazione del sistema, questa può
essere trasformata per evidenziare particolari proprietà del sistema.

\newpage
\section{Costruzione di un modello ISU}
Si può costruire seguendo quattro passaggi
\begin{enumerate}
\item Scrivere tutte le equazioni del sistema (eq. diff.) conoscendo il modello fisico del sistema
\item Individuare le variabili d'ingresso e uscita, mediante il principio di causalità
\item Scegliere le variabili di stato dopo averne individuato il numero. Una scelta possibile è
quella mediante la regola di Cauchy
\item Riscrivere le equazioni di partenza trovate al punto uno nella seguente forma
\begin{equation}\left\{ \begin{aligned}
&\dot{x}_1 = f_1\left(x_1,\ldots,x_n,u_1,\ldots,u_m,t\right)\\
&\ \vdots \qquad \vdots\\
&\dot{x}_n = f_n\left(x_1,\ldots,x_n,u_1,\ldots,u_m,t\right)\\
&y_1 = g_1\left(x_1,\ldots,x_n,u_1,\ldots,u_m,t\right) \\
&\ \vdots \qquad \vdots\\
&y_n = g_n\left(x_1,\ldots,x_n,u_1,\ldots,u_m,t\right)
\end{aligned}\right.\qquad\text{ISU}
\label{eq.:ISU_generale}
\end{equation}
la $t$ in questo caso indica l'eventuale tempo varianza dei parametri del sistema e non la
dipendenza dal tempo delle variabili di stato (assunta vera implicitamente).
\end{enumerate}

\subsection{Costruzione modello ISU RLC}
Si considera l'esempio mostrato in figura \ref{Fig.:circuito_RLC}, analizzando l'equazione
\ref{eq.:ISU_generale} si vede che sono presenti $n$ equazioni differenziali di primo grado dato
che le funzioni $f$ e $g$ sono equazioni algebriche, non contengono termini differenziali.
Sono invece presenti $p$ equazioni puramente algebriche $g$ che legano lo stato, l'ingresso, il
tempo alle uscite.

\begin{enumerate}
\item Ricordando l'equazione del sistema ricavata in \ref{eq:equazione_RLC}
si formalizza il modello
\begin{equation}\left\{
\begin{aligned}
 e &= R i + L\dot{i} + \frac{q}{C}\\
 i &= \dot{q}
\end{aligned}\right.\end{equation}
\item Si individuano tutte le variabili: $e,i,q$

Si individuano dunque le variabili d'ingresso tra quelle con ordine di derivata più basso, in
questo caso la $e$ è differenziata zero volte, sarà un ingresso, in questo caso l'unico; le altre,
differenziate una volta, saranno le potenziali variabili di uscita.
\item Si scelgono le variabili di stato, si deve prima determinare l'ordine del sistema:\newline
\emph{il
numero di variabili di stato è pari alla somma del numero di volte che le variabili delle equazioni
compaiono differenziate tolte le variabili d'ingresso.}

Sommando l'ordine di derivazione di $i$ e $q$, variabili non d'ingresso si ottiene dunque $n=2$.

\item Si possono scegliere come variabili di stato tutte quelle che determinano le condizioni
iniziali, ossia\newline
\emph{tutte le variabili che compaiono differenziate e in numero pari al numero di volte
in cui viene differenziata.}

Per il sistema in esame
$$
x_1 = q,\quad x_2 = i
$$
\item Scrittura delle equazioni in forma ISU

$$
\left\{\begin{aligned}
\dot{x}_1 &= f_1\left(x_1,x_2,u\right) = \dot{q} = i = x_2\\
\dot{x}_2 &=f_2(x_1,x_2,u) = \dot{i} = \frac{e}{L} - \frac{R}{L}i-\frac{1}{LC}q =\\
&= -\frac{1}{LC}x_1 - \frac{R}{L}x_2 + \frac{1}{L}u\\
y_1 &= i = g_1(x_1,x_2,u)=x_2\\
y_2 &=q = g_2(x_1,x_2,u)=x_1
\end{aligned}\right.
$$
\end{enumerate}

\newpage
\section{Classificazione dei sistemi dinamici}
Per comodità si riporta la \ref{eq.:ISU_generale} in forma più compatta.
Si costruisce un vettore di variabili di stato, d'ingressi e di uscite
$$
x = \begin{bmatrix}
x_1\\
x_2\\
\vdots\\
x_n
\end{bmatrix} \in \mathbb{R}^n \textcolor{red}{\left(\in\mathbb{C}^n\right)}\
u=\begin{bmatrix}
u_1\\
u_2\\
\vdots\\
u_m
\end{bmatrix}\in\mathbb{R}^m\
y=\begin{bmatrix}
y_1\\
y_2\\
\vdots\\
y_p
\end{bmatrix}\in \mathbb{R}^p
$$
l'esistenza di variabili complesse per le variabili di stato è giustificata da scelte pratiche
utili alla risoluzione del problema anche se non legate strettamente a fenomeni fisici.

Di conseguenza la $f$ sarà un vettore di funzioni
$$\left.\begin{aligned}
&f_1\left(x_1,...,x_n,u_1,...,u_m,t\right)=f_1(x,u,t)\\
&\ \vdots\\
&f_n\left(x_1,\ldots,x_n,u_1,\ldots,u_m,t\right)=f_n(x,u,t)
\end{aligned}\right\} \Rightarrow f(x,u,t)=\begin{bmatrix}
f_1(x,u,t)\\
\vdots\\
f_n(x,u,t)
\end{bmatrix}
$$
Analogamente per le funzioni $g$
$$
\left.\begin{aligned}
&g_1(x,u,t)\\
&\ \vdots\\
&g_p(x,u,t)
\end{aligned}\right\}\Rightarrow
g(x,u,t) = \begin{bmatrix}
g_1(x,u,t)\\
\vdots\\
g_p(x,u,t)
\end{bmatrix}
$$

La forma generale della ISU diventa
\begin{equation}\text{ISU }\left\{
\begin{aligned}
\dot{x} = f(x,u,t)\\
y = g(x,u,t)
\end{aligned}\right.
\label{eq.:ISU_compatta}
\end{equation}

Si suppone che le funzioni $f,g$ siano lineari in $x$ e $u$ ossia che possano essere espresse come
combinazione lineare di quelle variabili $(x,u)$.

Ad esempio la funzione $f$
$$\begin{aligned}
&\dot{x}_1 = a_{11}(t)x_1+a_{12}(t)x_2 + \ldots + a_{1n}(t)x_n + b_{11}(t)u_1+\ldots+b_{1m}(t)u_m\\
&\ \vdots\\
&\dot{x}_n = a_{n1}(t)x_1 +  a_{n2}(t)x_2 + \ldots + a_{nn}(t)x_n + b_{n1}(t)u_1+\ldots+b_{nm}(t)u_m
\end{aligned}
$$

Analogamente la funzione $g$
$$\begin{aligned}
&g_1 = c_{11}(t)x_1 + \ldots + c_{1n}(t)x_n + d_{11}(t)u_1 + \ldots + d_{1m}(t)u_m\\
&\ \vdots\\
&g_p = c_{p1}(t)x_1 + \ldots + c_{pn}(t)x_n + d_{p1}(t)u_1+\ldots+d_{pm}(t)u_m
\end{aligned}$$

Si può scrivere il sistema in forma vettoriale ottenuto dal prodotto di un vettore riga contenente
i coefficienti e un vettore colonna contenente le variabili, questo per ogni elemento di
$\dot{x}$, si ottiene dunque una matrice $n\times n$ di coefficienti ed un vettore colonna di
variabili $x$.
$$\begin{aligned}
f = \dot{x} = A \cdot x + B\cdot u
\end{aligned}
$$
con
$$
A = \begin{bmatrix}
a_{11} & \dots & a_{1n} \\
\vdots & \ddots & \vdots \\
a_{n1} & \dots & a_{nn}
\end{bmatrix} \in \mathbb{R}^{n\times n} \quad
B = \begin{bmatrix}
b_{11} & \dots & b_{1m} \\
\vdots & \ddots & \vdots \\
b_{n1} & \dots & b_{nm}
\end{bmatrix} \in \mathbb{R}^{n\times m}
$$
La matrice $A$ è sempre quadrata mentre la matrice $B$ ha una dimensione che dipende dal numero di
ingressi e può diventare un vettore se l'ingresso è unico.

Si ripete la procedura per le funzioni $g$
$$
g = y = C\cdot x + D\cdot u
$$
con
$$
C = \begin{bmatrix}
c_{11} & \dots & c_{1n} \\
\vdots & \ddots & \vdots \\
c_{p1} & \dots & c_{pn}
\end{bmatrix} \in \mathbb{R}^{p\times n} \quad
D = \begin{bmatrix}
d_{11} & \dots & d_{1m} \\
\vdots & \ddots & \vdots \\
d_{p1} & \dots & d_{pm}
\end{bmatrix} \in \mathbb{R}^{p\times m}
$$
La matrice $D$ diventa uno scalare se il sistema ha un solo ingresso e una sola uscita.

Riassumendo
\begin{equation}
\text{ISU } \left\{\begin{aligned}
\dot{x} &= A(t)x + B(t) u \\
y &= C(t) x + D(t) u
\end{aligned}\right.
\label{eq.:ISU_compatta_matrice}
\end{equation}
\emph{Un sistema si dice \textbf{lineare} se le funzioni $f$ e $g$ sono entrambe lineari nello stato
e nell'ingresso},
ossia se possono essere scritte nella forma \ref{eq.:ISU_compatta_matrice}.

Se le equazioni del sistema compaiono nella seguente forma
$$\left\{\begin{aligned}
\dot{x} &= f(x,u)\\
y &= g(x,u)\end{aligned}\right.
$$
allora il sistema si dice \textit{stazionario} o \textit{tempo invariante},
ossia il sistema avrà sempre la stessa evoluzione fissato l'ingresso e lo stato iniziale. In caso
contrario il sistema si comporterebbe in maniera differente a seconda di quando viene
sollecitato e analizzato.
Se il sistema gode di entrambe le proprietà si dirà \textit{Lineare Tempo Invariante} e saranno
quelli prevalentemente analizzati nel corso, sono sempre risolvibili.
\newpage
\subsection{Ulteriori classificazioni}
Nel seguente sistema, l'uscita non dipende \textit{direttamente} dall'ingresso
$$
\left\{\begin{aligned}
\dot{x} &= f(x,u,t) \\
y &= g(x,t)
\end{aligned}
\right.
$$
viene definito \emph{sistema strettamente proprio}, in caso contrario è \emph{proprio}.

Nel caso di sistemi lineari
$$
\left\{
\begin{aligned}
\dot{x} &= Ax + Bu \\
y &= Cx
\end{aligned}
\right. \qquad D = 0 \Leftrightarrow \text{Strettamente proprio}
$$
In natura esistono solo sistemi strettamente propri, non esistono quelli propri,
qualunque ingresso ha bisogno di propagarsi nel sistema entro un certo tempo, non può quindi
modificare l'uscita istantaneamente.
