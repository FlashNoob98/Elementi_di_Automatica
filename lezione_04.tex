%Lezione 04

\subsection{Amplificatore operazionale}

Si introduce il modello dell'amplificatore operazionale, rappresentato da un
triangolo con tre morsetti, due in ingresso ed uno in uscita, quello con il
segno $-$ viene detto invertente, quello con il segno $+$ è non invertente.
Solitamente sono anche indicati i morsetti di alimentazione.
\begin{figure}[H]
 \centering
 \begin{circuitikz}
  \draw (0,0) node[op amp, anchor=-](OA){}
 (OA.+) -- ++(-1,0)
 to [short, -*] (-1,-1);
 \draw
(OA.-) -- ++(-1,0)
 to [short, -*] (-1,0);
 \draw (-1,-1) to [open,v^>=$e$] (-1,0);
 \draw (OA.out) -- ++(0,0) to [open,v^<=$v$,*-*] ++(0,-1)
  node[ground]{} ;
  \draw (OA.up)  node[vcc]{$+V_{cc}$};
  \draw (OA.down) node[vee]{$-V_{cc}$};
 \end{circuitikz}
\end{figure}
L'equazione caratteristica è
$$
v = -Ae
$$
Solitamente il valore $A$ è molto grande, anche \SI{e6}{} così come l'impedenza
d'ingresso tra i morsetti $+$ e $-$.

Dato un segnale in ingresso dunque, questo verrà restituito in uscita invertito
ed amplificato. Questa condizione è verificata se si suppone che la $v$ è
compresa tra i valori $[-V_{cc},+V_{cc}]$ di alimentazione.
In caso contrario l'uscita verrebbe ``tagliata'' e ci sarebbe un comportamento
non lineare, ipotesi di lavoro sempre evitata in questo corso.

Di conseguenza il segnale d'ingresso dovrebbe essere di valori pari a $v/A$,
valori bassissimi, si può ritenere dunque che la caduta di tensione
all'ingresso sarà prossima a \SI{0}{\volt}, è quindi come se l'ingresso fosse
un corto circuito (anche se non lo è).

L'impedenza d'ingresso, essendo di ordine elevato, implica che anche la
corrente nello stadio d'ingresso sia prossima a \si{0}, quindi si può modellare
come un circuito aperto.

In conclusione lo stadio d'ingresso si comporta sia da circuito chiuso che
circuito aperto e viene chiamato \textbf{corto circuito virtuale}.

Lo stadio d'uscita è confrontabile con un generatore ideale di tensione.

Si consideri il seguente circuito:
\begin{figure}[H]
\centering
\begin{circuitikz}
\draw (3,-.5) node[op amp](op2){}
      (0,0) to [R=$R_1$,*-*]  (op2.-)
            to ++(0,1)
            to [R=$R_2$] (4,1);
\draw (op2.out) |- (4,1) ;
\draw (op2.out) to [short,-*] ++(1,0)
                to [open,-*,v^<=$v$] ++(0,-0.5)
                node[ground]{};
\draw (op2.+) to [short] ++(0,-.5)
        node[ground]{};
\draw (0,-1.5) node[ground]{} to [open,v^>=$e$,*-*] (0,0);
\end{circuitikz}
\end{figure}
