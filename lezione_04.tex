%Lezione 04

\subsection{Amplificatore operazionale}

Si introduce il modello dell'amplificatore operazionale, rappresentato da un
triangolo con tre morsetti, due in ingresso ed uno in uscita, quello con il
segno $-$ viene detto invertente, quello con il segno $+$ è non invertente.
Solitamente sono anche indicati i morsetti di alimentazione.
\begin{figure}[H]
 \centering
 \begin{circuitikz}
  \draw (0,0) node[op amp, anchor=-](OA){}
 (OA.+) -- ++(-1,0)
 to [short, -o] (-1,-1);
 \draw
(OA.-) -- ++(-1,0)
 to [short, -o] (-1,0);
 \draw (-1,-1) to [open,v^>=$e$] (-1,0);
 \draw (OA.out) -- ++(0,0) to [open,v^<=$v_{out}$] ++(0,-1)
  node[ground]{} ;
 \end{circuitikz}

\end{figure}
