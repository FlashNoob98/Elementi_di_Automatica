
\section{Modi naturali nel regime forzato}
Si analizza lo stato di un sistema forzato da un impulso
$$
x_f(t) = \left.\int_{0}^t e^{A(t-\tau)}Bu(\tau) d\tau = e^{At}Bc = H(t)c =
x_l(t)\right|_{x_0=Bc}
$$
Sia il forzamento impulsivo $ u(t)  = \delta(t)c$, $H(t)$ è la matrice delle
risposte impulsive dello stato.
Si ricorda l'equazione dell'evoluzione libera
$$
x_l(t) = e^{At}x_0
$$
Si nota che $x_0$ e $Bc$ hanno la stessa dimensione e sono entrambi costanti
quindi l'evoluzione forzata è pari all'evoluzione libera con condizioni
iniziali $x_0$ pari $Bc$, ossia lo stato è stato modificato istantaneamente, a
partire dall'origine, mediante la funzione impulso.
L'evoluzione libera sarà composta dai modi naturali ricavati nell'analisi
dell'evoluzione libera.

Si supponga che il vettore $c$ sia nullo ovunque eccetto per una componente $j$
$$
c^T= \begin{bmatrix}
0& \ldots & 0 & 1 & 0 & \ldots & 0
\end{bmatrix}
$$
La risposta forzata sarà
$$
x_f(t)=e^{at}Bc = e^{At}b_j = h_j(t)
$$
Si ottiene solo la colonna $j$-esima di $H(t)$ ossia la risposta all'impulso
all'ingresso $j$-esimo.

$$
H(t) = e^{At}B = \sum_j e^{\lambda_j t}u_jv_j^TB
$$
Il prodotto $v_j^TB$ è un vettore riga, $v_j^T$ è sicuramente un vettore non
nullo, $B$ allo stesso tempo è la matrice degli ingressi, non ha senso
considerarla nulla; ad ogni modo il prodotto tra questi due termini può
comunque dare origine al vettore nullo, se $v_j^T$ appartiene al \textit{nullo
sinistro} di $B$, o comunque essere nullo per alcuni autovalori $j$, se ciò
accade, il modo $j$ esimo non compare nel risultato, si parla di modo
\textit{non eccitabile} dall'ingresso, viceversa
$$
v_j^TB \neq 0 \Leftrightarrow \text{Modo $j$ eccitabile}
$$

Analisi della funzione $\Psi(t)$ dell'uscita
$$
\Psi(t) = Ce^{At} = \sum_j e^{\lambda_i t} Cu_jv_j^T
$$
Analogamente al caso precedente, il vettore $u_j$ può essere contenuto nel
\textit{kernel} di $C$, il prodotto sarà il vettore nullo, viceversa
$$
Cu_j\neq 0 \Leftrightarrow \text{Modo $j$ osservabile}
$$

L'uscita forzata, si rappresenta la matrice delle risposte all'impulso in uscita
$$
W(t) = Ce^{At}B = \sum_j e^{\lambda_j t} C u_jv_j^T B
$$
Il modo sarà nullo sia se non è osservabile, sia se non è eccitabile.

\subsection{Risposta al gradino}
Si valuta l'equazione dello stato
$$
x_f(t) = \int_0^t e^{A(t-\tau)}B\delta_{-1}(\tau)d\tau =
\int_0^t e^{A(t-\tau)} d\tau\cdot B
$$
Si ipotizza per semplicità che $A$ sia
diagonalizzabile
$$\begin{aligned}
x_f(t)&\stackrel{A\text{ diag.}}{=}
U\int_0^t e^{\Lambda(t-\tau)}d\tau U^{-1} B = U
\int_0^t\text{diag}\left\{e^{\lambda_1(t-\tau)}
,\ldots,e^{\lambda_n(t-\tau)}
\right\} d \tau U^{-1}B\\
&= U\text{diag}\left\{
\frac{e^{\lambda_1t}-1}{\lambda_1},\ldots,\frac{e^{\lambda_nt}-1}{\lambda_n}
\right\}U^{-1}B
\end{aligned}$$

Il prodotto $U^{-1}B$ è pari a
$$
U^{-1}B = \begin{bmatrix}
           v_1^T \\
           \vdots \\
           v_n^T
          \end{bmatrix}B
$$
Sarà dunque nullo per alcune righe, ossia per i modi non eccitabili, dunque non
comparirà il modo $j$-esimo in uscita.

La risposta forzata nell'uscita, con ipotesi semplificativa di sistema
strettamente proprio $(D=0)$
$$
y_f(t) = Cx_f(t) = CU\text{diag}\left\{
\frac{e^{\lambda_1t}-1}{\lambda_1},\ldots,\frac{e^{\lambda_nt}-1}{\lambda_n}
\right\}U^{-1}B
$$
Alcune colonne del prodotto $CU$ si potranno annullare, individuano i modi non
osservabili mentre le righe nulle a destra $U^{-1}B$ rappresentano i modi non
eccitabili, dunque i modi in uscita devono essere sia osservabili che
eccitabili.

Si può eseguire l'estensione degli estremi di integrazione dato che
$u(t)=0,H(t)=0\forall t<0$ e
$$
x_f(t) = \int_0^t H(t-\tau) u(\tau) d\tau = \int_{-\infty}^{\infty} H(t-\tau)
u(\tau)d\tau \stackrel{\text{def}}{=} H(t)*u(t)
$$
