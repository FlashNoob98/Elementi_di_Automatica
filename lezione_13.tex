
\section{Modi naturali nel regime forzato}
Si analizza lo stato di un sistema forzato da un impulso
$$
x_f(t) = \left.\int_{0}^t e^{A(t-\tau)}Bu(\tau) d\tau = e^{At}Bc = H(t)c =
x_l(t)\right|_{x_0=Bc}
$$
Sia il forzamento impulsivo $ u(t)  = \delta(t)c$, $H(t)$ è la matrice delle
risposte impulsive dello stato.
Si ricorda l'equazione dell'evoluzione libera
$$
x_l(t) = e^{At}x_0
$$
Si nota che $x_0$ e $Bc$ hanno la stessa dimensione e sono entrambi costanti
quindi l'evoluzione forzata è pari all'evoluzione libera con condizioni
iniziali $x_0$ pari $Bc$, ossia lo stato è stato modificato istantaneamente, a
partire dall'origine, mediante la funzione impulso.
L'evoluzione libera sarà composta dai modi naturali ricavati nell'analisi
dell'evoluzione libera.

Si supponga che il vettore $c$ sia nullo ovunque eccetto per una componente $j$
$$
c^T= \begin{bmatrix}
0& \ldots & 0 & 1 & 0 & \ldots & 0
\end{bmatrix}
$$
La risposta forzata sarà
$$
x_f(t)=e^{at}Bc = e^{At}b_j = h_j(t)
$$
Si ottiene solo la colonna $j$-esima di $H(t)$ ossia la risposta all'impulso
all'ingresso $j$-esimo.

$$
H(t) = e^{At}B = \sum_j e^{\lambda_j t}u_jv_j^TB
$$
Il prodotto $v_j^TB$ è un vettore riga, $v_j^T$ è sicuramente un vettore non
nullo, $B$ allo stesso tempo è la matrice degli ingressi, non ha senso
considerarla nulla; ad ogni modo il prodotto tra questi due termini può
comunque dare origine al vettore nullo, se $v_j^T$ appartiene al \textit{nullo
sinistro} di $B$, o comunque essere nullo per alcuni autovalori $j$, se ciò
accade, il modo $j$ esimo non compare nel risultato, si parla di modo
\textit{non eccitabile} dall'ingresso, viceversa
$$
v_j^TB \neq 0 \Leftrightarrow \text{Modo $j$ eccitabile}
$$

Analisi della funzione $\Psi(t)$ dell'uscita
$$
\Psi(t) = Ce^{At} = \sum_j e^{\lambda_i t} Cu_jv_j^T
$$
Analogamente al caso precedente, il vettore $u_j$ può essere contenuto nel
\textit{kernel} di $C$, il prodotto sarà il vettore nullo, viceversa
$$
Cu_j\neq 0 \Leftrightarrow \text{Modo $j$ osservabile}
$$

L'uscita forzata, si rappresenta la matrice delle risposte all'impulso in uscita
$$
W(t) = Ce^{At}B = \sum_j e^{\lambda_j t} C u_jv_j^T B
$$
Il modo sarà nullo sia se non è osservabile, sia se non è eccitabile.

\subsection{Risposta al gradino}
Si valuta l'equazione dello stato
$$
x_f(t) = \int_0^t e^{A(t-\tau)}B\delta_{-1}(\tau)d\tau =
\int_0^t e^{A(t-\tau)} d\tau\cdot B
$$
Si ipotizza per semplicità che $A$ sia
diagonalizzabile
$$\begin{aligned}
x_f(t)&\stackrel{A\text{ diag.}}{=}
U\int_0^t e^{\Lambda(t-\tau)}d\tau U^{-1} B = U
\int_0^t\text{diag}\left\{e^{\lambda_1(t-\tau)}
,\ldots,e^{\lambda_n(t-\tau)}
\right\} d \tau U^{-1}B\\
&= U\text{diag}\left\{
\frac{e^{\lambda_1t}-1}{\lambda_1},\ldots,\frac{e^{\lambda_nt}-1}{\lambda_n}
\right\}U^{-1}B
\end{aligned}$$

Il prodotto $U^{-1}B$ è pari a
$$
U^{-1}B = \begin{bmatrix}
           v_1^T \\
           \vdots \\
           v_n^T
          \end{bmatrix}B
$$
Sarà dunque nullo per alcune righe, ossia per i modi non eccitabili, dunque non
comparirà il modo $j$-esimo in uscita.

La risposta forzata nell'uscita, con ipotesi semplificativa di sistema
strettamente proprio $(D=0)$
$$
y_f(t) = Cx_f(t) = CU\text{diag}\left\{
\frac{e^{\lambda_1t}-1}{\lambda_1},\ldots,\frac{e^{\lambda_nt}-1}{\lambda_n}
\right\}U^{-1}B
$$
Alcune colonne del prodotto $CU$ si potranno annullare, individuano i modi non
osservabili mentre le righe nulle a destra $U^{-1}B$ rappresentano i modi non
eccitabili, dunque i modi in uscita devono essere sia osservabili che
eccitabili.

Si può eseguire l'estensione degli estremi di integrazione dato che
$u(t)=0\ \forall t<0$ e $H(t) = 0\ \forall t>0$
si ottiene la definizione di \textbf{prodotto di convoluzione}, la matrice $H$
contiene dunque solo i modi eccitabili.
$$
x_f(t) = \int_0^t H(t-\tau) u(\tau) d\tau = \int_{-\infty}^{\infty} H(t-\tau)
u(\tau)d\tau \stackrel{\text{def}}{=} H(t)*u(t)
$$

Analogamente per la $y_f(t)$

$$
y_f(t) \stackrel{\text{def}}{=} \int_0^t W(t-\tau)u(\tau)d\tau =
\int_{-\infty}^{+\infty}W(t-\tau)u(\tau)d\tau \stackrel{\text{def}}{=}W(t)*u(t)
$$

%\chapter{Analisi nel dominio di Laplace}
\newpage
\section{Trasformata di Laplace}
Sia la funzione $f$ reale e nulla per $t<0$, la seconda ipotesi permette di
ricavare univocamente l'antitrasformata, non è necessaria per eseguire la
trasformata.
$$
f:\mathbb{R}\to\mathbb{R},\ f(t)=0\ \forall t <0
$$
Viceversa se la funzione non è nulla per $t<0$
potrebbe comunque esserlo per un certo istante minore di zero; se il sistema è
tempo invariante si può effettuare una traslazione temporale, prima e dopo aver
trasformato e antitrasformato.
Viceversa una funzione sinusoidale non può essere trasformata nel dominio di
laplace e antritrasformata.

\textbf{Definizione di trasformata}, l'integrale parte da $0^-$ per
sottolineare che se sono presenti impulsi all'istante iniziale, questi vanno
considerati
$$
F(s)=\Lap[f(t)] = \int_{0^-}^{+\infty} f(\tau)e^{-s\tau} d\tau \qquad s
\in \mathbb{C}
$$
Non è detto che l'integrale converga ma se converge per un certo $s$ con
parte reale $\sigma$ allora convergerà per tutti gli altri $s$ con parte
reale maggiore di $\sigma$, si individua quindi un semipiano destro di
convergenza.

Dato che la trasformata di Laplace è una funzione \textit{analitica} si può
estendere la definizione a qualunque punto del semipiano di convergenza eccetto
le singolarità polari della funzione.

\subsection{Proprietà della trasformata}
\begin{itemize}
 \item \textbf{Linearità:} $
  \Lap [\alpha f_1(t) + \beta f_2(t)] = \alpha F_1(s) + \beta F_2(s)
 $
 \item \textbf{Traslazione in $t$:} $
 \Lap[ f(t-\tau)  ] = e^{-s\tau}F(s) \quad \tau>0
 $
 \item \textbf{Traslazione in $s$:} $
 \Lap[e^{\sigma t}f(t)] = F(s-\sigma)
 $
 \item \textbf{Derivazione nel tempo:}
 $$\begin{aligned}
  \Lap[&\dot{f}(t)]  = sF(s) - f(0^{-})\\
 \Lap[&\ddot{f}(t)] = s\Lap[\dot f(t)] - \dot{f}(0^-) = s^2 F(s) - sf(0^-) -
\dot{f}(0^-)\\
&\vdots \qquad \text{iterando}\\
\Lap[&f^{(n)}(t)]= s^nF(s) - \sum_{i=1}^n s^{n-i}f^{(i-1)}(0^-)
 \end{aligned}$$
 \item \textbf{Derivazione in $s$:} $\Lap [tf(t)]=
 -\frac{dF(s)}{ds}
 $
 \item \textbf{Integrazione in $t$:} $
 \Lap[\int_0^tf(\tau)d\tau] = \frac{1}{s}F(s)
 $
 \item \textbf{Convoluzione in $t$:} $
 \Lap[f_1(t)*f_2(t)] = F_1(s)\cdot F_2(s)
 $
 \item \textbf{Teorema valore iniziale:} $$
 \lim_{t\to 0^+} f(t) = f(0^+) = \lim_{s\to +\infty} sF(s)
 $$
 \item \textbf{Teorema valore finale:} $$
 \lim_{t\to + \infty} f(t) = \lim_{s\to 0} s F(s)
 $$
\end{itemize}

\subsection{Trasformate degli ingressi canonici}
$$\begin{aligned}
\Lap[\delta(t)] &= 1 \quad
\Lap[\delta_{-1}(t)] = \frac{1}{s}\quad
\Lap[\delta_{-2}(t)] = \frac{1}{s^2}\quad
\Lap[\delta_{-3}(t)] = \frac{1}{s^3} \\
\Lap[\delta_{-n}(t)] &= \Lap\left[\frac{t^{n-1}}{(n-1)!}\delta_{-1}(t)\right] =
\frac{1}{s^n}\\
\Lap\left[e^{\alpha t}\delta_{-1}(t)\right] &= \frac{1}{s-\alpha}\\
\Lap\left[te^{\alpha t}\delta_{-1}(t)\right] &=
\Lap \left[e^{\alpha t}\delta_{-2}(t)\right] =
\frac{1}{(s-\alpha)^2}\\
\Lap\left[\frac{t^{n-1}}{(n-1)!}e^{\alpha t} \delta_{-1}(t)\right] &=
\frac{1}{(s-\alpha)^n} \\
\Lap\left[\sin(\omega t)\delta_{-1}(t)\right] &= \frac{\omega}{s^2+\omega^2},\
\Lap\left[\cos(\omega t)\delta_{-1}(t)\right] = \frac{s}{s^2+\omega^2}\\
\Lap\left[e^{\alpha t}\sin(\omega t) \delta_{-1}(t)\right] &=
\frac{\omega}{(s-\alpha)^2+\omega^2},\ \Lap\left[e^{\alpha t}\cos(\omega
t)\delta_{-1}(t)\right] = \frac{s-\alpha}{(s-\alpha)^2+\omega^2}\\
\Lap\left[t\sin(\omega t)\delta_{-1}(t)\right] & = \frac{2\omega
s}{(s^2+\omega^2)^2},\  \Lap\left[t\cos(\omega t)\delta_{-1}(t)\right] =
\frac{s^2-\omega^2}{(s^2+\omega^2)^2}\\
\Lap\left[te^{\alpha t}\sin(\omega t) \delta_{-1}(t)\right] & =
\frac{2\omega(s-\alpha)}{\left((s-\alpha)^2+\omega^2\right)^2},\
\Lap\left[te^{\alpha t}\cos(\omega t) \delta_{-1}(t)\right] =
\frac{(s-\alpha)^2-\omega^2}{\left((s-\alpha)^2+\omega^2\right)^2}
\end{aligned}
$$

\subsubsection{Trasformata di una funzione periodica}
Sia $f$ una funzione periodica di periodo $T$
$$
f_T: \mathbb{R}\to\mathbb{R}
$$
Se ne vuole calcolare la trasformata di Laplace
$$
F(s) = \Lap[f_T(t)] = \left(\int_0^Tf(t)e^{-st}dt\right)\cdot\frac{1}{1-e^{-sT}}
$$

%\newpage
\subsubsection{Antitrasformata razionali fratte}
Sia la seguente funzione fratta con grado $\frac{m}{n}$
$$
F(s) = \frac{N(s)}{D(s)}
$$
Le radici del numeratore vengono chiamate \textbf{zeri}, le radici del
denominatore vengono denominate \textbf{poli}. La funzione è ovviamente non
definita sui poli.
La differenza $n-m$ tra il numero di poli e zeri, chiamata anche \textit{eccesso
poli-zeri} viene detta \textbf{grado relativo} della funzione.

$$
n-m =
\left\{\begin{aligned}
&n>m\\
&n=m\\
&n<m  \rightarrow \quad \text{Funzioni non causali}
\end{aligned}\right.
$$
I primi due casi saranno gli unici considerati in questo corso.

\textbf{n=m}
$$
F(s) = \frac{N(s)}{D(s)} = \frac{N'(s)}{D(s)} + d\qquad \frac{[n-1]}{[n]} +d
$$
Si ottiene una funzione con grado del numeratore pari a quello del denominatore
meno uno sommata ad una costante $d$.

\textbf{Esempio numerico}
$$
F(s) =\frac{3s^2+s+2}{s+1}  \rightarrow
\polylongdiv[vars=s,style=D]{3*s^2+s+2}{s+1}
$$
Il quoziente $d$ sarà $3s-2$ e il resto $N'$ sarà $4$, la funzione dunque si
riscrive come
$$
F(s) = \frac{4}{s+1} + 3s - 2
$$
L'antitrasformata sarà
$$
\Lap^{-1}[F(s)] = f(t) = 4e^{-t}+3\dot{\delta}(t)-2\delta(t)
$$
Non ha molto senso fisico la notazione di $\dot{\delta}(t)$

\newpage
\textbf{n$>$m}\newline
Si sviluppa la funzione in fratti semplici
$$
F(s) = \frac{N(s)}{D(s)} = \frac{R_1}{s-p_1} + \frac{R_2}{s-p_2} + \dots +
\frac{R_n}{s-p_n}
$$
I termini $R_i$ sono chiamati residui polari associati al polo $p_i$, si
supponga per il momento che i poli siano tutti distinti tra loro.
$$
R_i = \lim_{s\to p_i} (s-p_i)F(s)
$$
Si dimostra inoltre che
$$
\sum_{i=1}^n R_i = \left\langle
\begin{aligned}
0 &\qquad n-m>1\\
\frac{b_m}{a_n} &\qquad n-m=1
\end{aligned}\right.
$$
I coefficienti $a_n$ e $b_n$ si ottengono dai polinomi del numeratore e
denominatore:
$$
\left\{\begin{aligned}
N(s) &= b_ns^n + \ldots+b_0\\
D(s) &= a_n s^n + \ldots + a_0
\end{aligned}\right.
$$

\textbf{Esempio con poli coincidenti}\newline
Si supponga che il polo $p_j$ abbia molteplicità $3$
$$
F(s) =
\frac{R_1}{s-p_1}+\frac{R_2}{s-p_2}+\ldots+\frac{R_j}{s-p_j}+
\frac{R_j^{(2)}}{(s-p_j)^2 }+
\frac{R_j^{(3)}}{(s-p_j)^3}+\ldots+
\frac{R_n}{s-p_n}
$$
Il residuo associato alla potenza $k$-esima del polo $i$-esimo con molteplicità
$h$ si calcola con la seguente
$$
R_i^{(k)} = \lim_{s\to p_i} \frac{1}{(h-k)!}\frac{d^{h-k}}{ds^{h-k}}
\left[(s-p_i)^hF(s)\right]
$$
