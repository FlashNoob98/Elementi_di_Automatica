
\section{Risposta ai segnali canonici}
Ci si riferirà principalmente alla risposta al gradino e all'impulso, facendo
uso della funzione di trasferimento presentata nella sezione precedente.
Si esegue inizialmente un'analisi di tipo ``asintotico'', per $t\to \infty$,
successivamente per $t\to 0$. Questo tipo di risposta non richiede l'analisi
puntuale delle equazioni del sistema, è necessario analizzare alcune
caratteristiche della funzione di trasferimento.

La risposta all'impulso è per definizione l'antitrasformata della funzione di
trasferimento
$$
W(t) = \Lap^{-1} [W(s)]
$$
Sviluppati i fratti, la funzione avrà la seguente forma
$$
W(t) = \Lap^{-1}\left[\frac{R_0}{s} + \frac{R_1}{s^2} +\ldots +
\frac{R_{g-1}}{s^g} + \frac{p_1}{s-p_1} + \frac{p_2}{s-p_2} + \ldots \right]
$$
I primi fratti (fino a $R_{g-1}$) sono associati ai poli nell'origine, seguono
i fratti associati ai poli reali ($p_i$) o eventualmente complessi e coniugati.

Il valore di $g$ permette di capire il tipo di sistema e se positivo, il numero
di poli nell'origine.
L'antitrasformata del primo termine è un gradino, del secondo è una retta, poi
una parabola e così via.
Tutti questi segnali (eccetto il primo che resta costante) divergeranno per
$t\to \infty$.

Si analizzano gli altri poli, se ne esiste almeno uno con parte reale positiva,
si avrà un esponenziale divergente nell'antitrasformata, se invece sono tutti a
parte reale negativa la loro somma convergerà.
$$\left\{
\begin{aligned}
\exists p_i &: \Re(p_i) > 0 \Rightarrow \text{Divergono} \\
\forall p_i &: \Re(p_i) < 0 \Rightarrow \text{Convergono} \\
\forall p_i &: \Re(p_i) < 0 \text{ e qualcuno sull'asse } \Im\ (m_a=1)
\end{aligned}\right.
$$
Un terzo caso prevede la presenza di poli sull'asse immaginario, a parte reale
nulla 11:07
