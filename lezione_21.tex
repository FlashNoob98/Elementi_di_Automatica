
\section{Diagrammi di Nyquist}
Sono una forma di rappresentazione della risposta armonica, definiti sul piano
di Gauss, si ottiene una curva \textit{graduata} in $\omega$, ossia per ogni
punto $\omega$ corrisponderà una coppia di punti sul piano di Gauss.

Il modulo e la fase della funzione $W(j\omega)$ sono quelli ricavati
nell'analisi per i diagrammi di Bode.

Si potrebbe costruire il grafico tabellando tutti i valori per ogni $\omega$,
in alternativa si arriva alla costruzione a partire dai diagrammi di Bode.
Seguono le regole di tracciamento
\begin{enumerate}
 \item Costruire i diagrammi di Bode (con le eventuali correzioni)
 \item $g=0 \Rightarrow W(j0) \in \mathbb{R}$ si partirà da un punto sull'asse
reale di valore $K_B$
\item $W(j\omega)$ abbandona l'asse reale sempre ortogonalmente.
\item $g<0:W(j0) = 0$ il diagramma parte dall'origine \textit{oppure}

$g>0: |W(j0)| = +\infty$ Asintoto dipendente dalla fase iniziale

\item $n-m>0 \Rightarrow W(j\infty) = 0$ Sistema strettamente proprio, la
tangente dipenderà ancora dalla fase $\phase{W(j\infty)}$
\end{enumerate}

Si consideri un sistema del primo ordine
$$
W(s)  = \frac{K_B}{1+s\tau} \qquad
\left\{\begin{aligned}
\tau&>0\\
K_B&>0
\end{aligned}\right.
$$
Facendo riferimento alle figure \ref{fig.amplitude_binomio} per il modulo e
\ref{fig.phase_monomio} per la fase si partirà da un punto sull'asse reale
$K_B$ dato che la fase iniziale è nulla mentre si termina nell'origine con
angolo asintoticamente pari a \SI{-90}{\degree}
\begin{figure}[h]
\centering
\def\KB{3}
\def\TAU{0.5}
\begin{tikzpicture}[
gnuplot def/.append style={prefix={tikz/}}
]
\begin{scope}
\tikzset{
Nyquist grid/.style={black},
}
\NyquistGraph[smooth,samples=81]{-2:4}
{\POAmp{\KB}{\TAU}}{\POArg{\KB}{\TAU}}

\NyquistGrid
\end{scope}

\end{tikzpicture}
\caption{$K_B = \KB,\ \tau=\TAU$}
\end{figure}

La distribuzione di punti non è uniforme lungo la curva, sono in realtà tutti
addensati nel punto iniziale e nell'origine per $\omega\ll\omega_H$ e
$\omega\gg\omega_H$.

Se $K_B$ fosse negativa con $\tau$ positiva si avrebbe il grafico ribaltato nel
secondo quadrante.

\subsubsection{Funzione del secondo ordine}
Si consideri una funzione del secondo ordine
$$
W(s) = \frac{K_B}{1+\frac{2\zeta}{\omega_n}s + \frac{s^2}{\omega_n^2}}
$$
Le figure d'esempio saranno questa volta la
\ref{fig.amplitude_trinomio} per quanto riguarda il modulo e la
\ref{fig.phase_trinomio} per la fase.
\begin{figure}[h]
\centering
\def\KB{1}
\def\ZETA{0.9}
\def\ZETAA{0.2}
\begin{tikzpicture}[
gnuplot def/.append style={prefix={tikz/}}
]
\begin{scope}
\tikzset{
Nyquist grid/.style={black},
}
\NyquistGraph[smooth,samples=81]{-2:4}
{\SOAmp{\KB}{\ZETA}{1}}{\SOArg{\KB}{\ZETA}{1}}

\NyquistGraph[smooth,samples=1250,color=red]{-2:4}
{\SOAmp{\KB}{\ZETAA}{1}}{\SOArg{\KB}{\ZETAA}{1}}

\NyquistGrid
\end{scope}

\end{tikzpicture}
\caption{$\textcolor{blue}{\zeta=\ZETA},\ \textcolor{red}{\zeta=\ZETAA}$}
\end{figure}
34:51
