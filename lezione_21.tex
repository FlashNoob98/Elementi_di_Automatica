
\section{Diagrammi di Nyquist}
Sono una forma di rappresentazione della risposta armonica, definiti sul piano
di Gauss, si ottiene una curva \textit{graduata} in $\omega$, ossia per ogni
punto $\omega$ corrisponderà una coppia di punti sul piano di Gauss.

Il modulo e la fase della funzione $W(j\omega)$ sono quelli ricavati
nell'analisi per i diagrammi di Bode.

Si potrebbe costruire il grafico tabellando tutti i valori per ogni $\omega$,
in alternativa si arriva alla costruzione a partire dai diagrammi di Bode.
Seguono le regole di tracciamento
\begin{enumerate}
 \item Costruire i diagrammi di Bode (con le eventuali correzioni)
 \item $g=0 \Rightarrow W(j0) \in \mathbb{R}$ si partirà da un punto sull'asse
reale di valore $K_B$
\item $W(j\omega)$ abbandona l'asse reale sempre ortogonalmente.
\item $g<0:W(j0) = 0$ il diagramma parte dall'origine \textit{oppure}

$g>0: |W(j0)| = +\infty$ Asintoto dipendente dalla fase iniziale

\item $n-m>0 \Rightarrow W(j\infty) = 0$ Sistema strettamente proprio, la
tangente dipenderà ancora dalla fase $\phase{W(j\infty)}$
\end{enumerate}

Si consideri un sistema del primo ordine
$$
W(s)  = \frac{K_B}{1+s\tau} \qquad
\left\{\begin{aligned}
\tau&>0\\
K_B&>0
\end{aligned}\right.
$$
