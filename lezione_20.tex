\subsection{Diagramma della fase}
Per presentare il diagramma della fase si riporta la funzione di trasferimento
nella forma di Bode
$$
W(s) = \frac{K_B}{(s)^g}\cdot \frac{\prod_i (1+s \tau_{zi})
\prod_i \left(1+\frac{s^2}{\omega_{nz_i}^2} +
\frac{2\zeta_{zi}}{\omega_{nz_i}}s\right) }
{\prod_i (1+s \tau_{pi})
\prod_i \left(1+\frac{s^2}{\omega_{np_i}^2} +
\frac{2\zeta_{pi}}{\omega_{np_i}}s\right) }
$$
Si studia la funzione sull'asse immaginario ossia quando $s=j\omega$
$$
W({j\omega}) = \frac{K_B}{({j\omega})^g}\cdot \frac{\prod_i (1+{j\omega}
\tau_{zi})
\prod_i \left(1-\frac{\omega^2}{\omega_{nz_i}^2} +
\frac{2\zeta_{zi}}{\omega_{nz_i}}\omega j\right) }
{\prod_i (1+{j\omega} \tau_{pi})
\prod_i \left(1-\frac{\omega^2}{\omega_{np_i}^2} +
\frac{2\zeta_{pi}}{\omega_{np_i}}\omega j\right) }$$

Si ricordano le proprietà della fase: la fase del prodotto è la somma delle
fasi, la fase del rapporto è la differenza delle fasi.

$$\begin{aligned}
\phase{W(j\omega)} &= \phase{K_B} - g\phase{j\omega} +
\sum_i\phase{1+j\omega\tau_{z_i}} +
\sum_i\phase{1-\frac{\omega^2}{\omega_{nz_i}^2}+\frac{2\zeta_{zi}}{\omega_{nz_i}
}\omega j} - \\
&-\sum_i\phase{1+j\omega\tau_{p_i}} -
\sum_i\phase{1-\frac{\omega^2}{\omega_{np_i}^2} +
\frac{2\zeta_{pi}}{\omega_{np_i}}\omega j}
\end{aligned}$$
Anche per la fase si può fare prima un diagramma asintotico e poi applicare le
correzioni.
$$
W_a(j\omega) = K_B,\ W_b(j\omega) = \frac{1}{j\omega},\
W_c(j\omega)=\frac{1}{1+j\omega_\tau},\ W_d(j\omega) =
\frac{1}{1-\frac{\omega^2}{\omega_n^2}+j\frac{2\zeta}{\omega_n}\omega}
$$

\textbf{Fase del guadagno} \`E un numero reale dunque la sua fase può essere 0
o -180\textdegree
$$
\phase{W_a(j\omega)} = \phase{K_B} \left\langle
\begin{aligned}
& 0 & & K_b > 0 \\
& \SI{-180}{\degree} & & K_B <0
\end{aligned}\right.
$$

\begin{figure}[h]
\centering
\begin{tikzpicture}[
gnuplot def/.append style={prefix={tikz/}}]
\begin{scope}[xscale=7/3,yscale=3/60]
\tikzset{
semilog lines/.style={black},
}
\UniteDegre
\semilog{-1}{2}{-40}{20}
\BodeGraph[asymp
lines,samples=400]{-1:2}{\IntAmp{-0.625}-\POAmpAsymp{1}{0.1}-\POAmpAsymp{1}{1}
+\SOAmpAsymp{1}{0.125}{4}}
\BodeGraph[samples=400]{-1:2}{\IntAmp{-0.625}-\POAmp{1}{0.1}-\POAmp{1}{1}
 +\SOAmp{1}{0.125}{4}}
\end{scope}
\end{tikzpicture}
%\caption{$\textcolor{red}{\zeta=\zOne},\quad \textcolor{blue}{\zeta=\zTwo} $}
\end{figure}
