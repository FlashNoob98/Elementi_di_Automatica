
\section{Regime con specifici segnali di ingresso}
\subsubsection{Ingresso esponenziale}
Si ha il segnale in ingresso
$$
u(t) = e^{\lambda t}\qquad \lambda\in\mathbb{C}
$$
La risposta a regime, se esiste è pari a
$$
y_r(t) = \int_0^{+\infty} W(\zeta) e^{\lambda(t-\zeta)} d\zeta = e^{\lambda t}
\int_0^{+\infty} W(\zeta)e^{-\lambda \zeta}
d\zeta
$$
l'integrale ottenuto è proprio la trasformata di Laplace della risposta
all'impulso $W(t)$ calcolata in $\lambda$. Per un sistema asintoticamente
stabile in forma minima, se l'ingresso è un esponenziale, l'uscita sarà
$$
y_r(t) = e^{\lambda t} W(\lambda)
$$
con $\lambda \neq \lambda_i$ autovalori della matrice della dinamica.
$W(\lambda)$ è dunque una costante, una funzione razionale fratta valutata in
$\lambda$, di conseguenza l'uscita sarà pari all'ingresso scalato per la
suddetta costante.
Questa condizione è verificata quasi sempre, ossia le risposte tendono a
seguire gli ingressi a regime.

Se $\lambda$ fosse però pari proprio ad uno zero della funzione di
trasferimento allora l'uscita sarebbe nulla per ogni istante di tempo.
Si dice che gli zeri reali hanno una proprietà \textit{bloccante} per gli
ingressi esponenziali.

\subsubsection{Ingresso sinusoidale}
Sia l'ingresso sinusoidale, sfruttando l'identità di Eulero e la linearità del
sistema, combinate al risultato precedentemente ottenuto per la funzione
esponenziale, si ha:
\begin{equation}
u(t) = \sin(\omega t) \stackrel{\text{Eulero}}{=} \frac{e^{j\omega
t}-e^{-j\omega t}}{2j}\stackrel{\text{Linearità}}{\longrightarrow} y_r(t) =
\frac{W(j\omega)e^{j\omega t} - W(-j\omega)e^{-j\omega t}}{2j}
\label{eq:risposta_sinusoidale}
\end{equation}
Va verificata l'esistenza delle funzioni di trasferimento, l'ascissa di
convergenza coincide con la singolarità polare a parte reale massima ma il
sistema è supposto asintoticamente stabile, di conseguenza tutti gli autovalori
avranno parte reale negativa, il semipiano di convergenza conterrà sicuramente
l'asse immaginario.

Si analizza la funzione di trasferimento valutata in $j\omega$ sfruttando la
definizione di trasformata di Laplace e ancora una volta la formula di Eulero:
$$
W(j\omega) = \int_0^{+\infty} W(t) e^{-j\omega t} dt = \int_0^{+\infty} W(t)
\left(\cos(\omega t) - j\sin(\omega t) \right)dt
$$
Dall'integrale si ottiene che
$$\left.\begin{aligned}
&\Re[W(j\omega)] \qquad \text{è pari}\\
&\Im[W(j\omega)] \qquad \text{è dispari}
\end{aligned}\right\}
\Rightarrow\left\{\begin{aligned}
&|W(j\omega)| \qquad \text{è pari} \\
&\phase{W(j\omega)}\qquad \text{è dispari}
\end{aligned}\right. \Rightarrow
W(-j\omega) = W^*(j\omega)
$$

Si rielabora il risultato \ref{eq:risposta_sinusoidale} con la proprietà appena
ottenuta.
$$\begin{aligned}
y_r(t) &= \frac{|W(j\omega)|e^{j\angle{(W(j\omega))}}e^{j\omega t}
- |W(j\omega)|e^{-j\angle({W(j\omega)})}e^{-j\omega t}  }
{2j} =\\
&=
|W(j\omega)|\frac{e^{j(\omega t + \angle(W(j\omega)))} - e^{-j(\omega t +
\angle(W(j\omega)))}}
{2j} = \\
&= |W(j\omega)|\sin\left(\omega t + \phase{W(j\omega)}\right)
\end{aligned}$$

Si riassume il risultato nel \textbf{Teorema della risposta armonica}
$$
u(t) = A\sin(\omega t + \varphi) \longrightarrow y_r(t) =
A|W(j\omega)|\sin\left( \omega t + \phase{W(j\omega)} + \varphi \right)
$$
Per un sistema asintoticamente stabile in forma minima, la risposta a regime di
un ingresso sinusoidale è ancora una sinusoide alla stessa frequenza di quella
in ingresso, amplificata o attenuata di una quantità pari al modulo della
restrizione all'asse immaginario della funzione di trasferimento, sfasata
inoltre di una quantità pari all'argomento della funzione $W(j\omega)$.

La funzione di trasferimento ristretta all'asse immaginario prende il nome di
\textit{risposta armonica} del sistema
$$
W(j\omega) = \left.W(s)\right|_{s=j\omega} = C(j\omega I-A)^{-1}B + D
$$
dà informazioni sul comportamento in frequenza del sistema, ampiezza e fase
dell'uscita dipenderanno dalla specifica frequenza in ingresso.

Potrebbe accadere che l'armonica in uscita sia esattamente nulla, quando
$j\omega$ è uno zero della funzione di trasferimento, in realtà due zeri
complessi e coniugati sull'asse immaginario.
Essendo la funzione $W(j\omega)$ una funzione regolare, l'intorno degli zeri
viene comunque attenuato dal sistema.

30:10
