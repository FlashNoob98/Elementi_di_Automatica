
\section{Regime con specifici segnali di ingresso}
\subsection{Ingresso esponenziale}
Si ha il segnale in ingresso
$$
u(t) = e^{\lambda t}\qquad \lambda\in\mathbb{C}
$$
La risposta a regime, se esiste è pari a
$$
y_r(t) = \int_0^{+\infty} W(\zeta) e^{\lambda(t-\zeta)} d\zeta = e^{\lambda t}
\int_0^{+\infty} W(\zeta)e^{-\lambda \zeta}
d\zeta
$$
l'integrale ottenuto è proprio la trasformata di Laplace della risposta
all'impulso $W(t)$ calcolata in $\lambda$. Per un sistema asintoticamente
stabile in forma minima, se l'ingresso è un esponenziale, l'uscita sarà
$$
y_r(t) = e^{\lambda t} W(\lambda)
$$
con $\lambda \neq \lambda_i$ autovalori della matrice della dinamica.
$W(\lambda)$ è dunque una costante, una funzione razionale fratta valutata in
$\lambda$, di conseguenza l'uscita sarà pari all'ingresso scalato per la
suddetta costante.


