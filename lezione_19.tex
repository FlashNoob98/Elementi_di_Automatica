
\section{Diagrammi di Bode}
I \textit{diagrammi di Bode} sono strumenti grafici utili a rappresentare la
funzione di risposta armonica $W(j\omega)$ che permette di descrivere il
comportamento in frequenza dei sistemi dinamici.

I diagrammi di Bode si dividono in due grafici, uno per rappresentare il modulo
di $W(j\omega)$, l'altro per la fase e prendono rispettivamente il nome di
diagramma del modulo e diagramma di fase.

Spesso l'asse delle ascisse è rappresentato in scala logaritmica, data
l'elevata ampiezza del dominio delle funzioni, di conseguenza il rapporto tra
due pulsazioni equidistanziate è sempre pari a dieci.
Questa distanza viene chiamata \textit{decade}.

L'asse lineare viene invece rappresentato in scala lineare ma i valori
riportati sono espressi in \textit{deciBel}
$$
|W(j\omega)|_{\si{\deci\bel}} = 20\log_{10}|W(j\omega)|
$$
Si usa per comodità il fattore $20$ e non $10$ come da definizione.
Il grafico prende il nome di scala semilogaritmica.
\begin{figure}[h]
\centering
\begin{tikzpicture}[
yscale=2/30,xscale=7/5
]
\tikzset{
semilog lines/.style={black},
}
\semilog{-2}{3}{-20}{10}
\end{tikzpicture}
\end{figure}

Il diagramma della fase è anch'esso rappresentato in scala logaritmico e la
fase viene rappresentata in scala lineare, in gradi o radianti.

Uno strumento rapido per rappresentare i diagrammi di Bode è quello di
utilizzare i \textit{diagrammi asintotici}, una rappresentazione approssimata
ma molto rapida dei sistemi.
Conviene innanzitutto trasformare la funzione di trasferimento nella forma di
Bode
$$
W(j\omega) = \frac{K_B}{(j\omega)^g}\cdot \frac{\prod_i (1+j\omega
\tau_{zi})
\prod_i \left(1-\frac{\omega^2}{\omega_{nz_i}^2} +
\frac{2\zeta_{zi}}{\omega_{nz_i}}\omega j\right) }
{\prod_i (1+j\omega \tau_{pi})
\prod_i \left(1-\frac{\omega^2}{\omega_{np_i}^2} +
\frac{2\zeta_{pi}}{\omega_{np_i}}\omega j\right) }
$$

Si vuole tracciare il modulo in deciBel
$$
|W(j\omega)|_{\si{\deci\bel}} = 20\log_{10}|W(j\omega)|
$$
Sfruttando le proprietà del logaritmo
$$\begin{aligned}
|W(j\omega)|_{\si{\deci\bel}} &= 20\log_{10}|K_B|-20g\cdot\log_{10}|j\omega| +
\sum_i 20\log_{10}|1+j\omega\tau_{zi}| + \\
&+\sum_i 20 \log_{10}
\left|1-\frac{\omega^2}{\omega_{nz_i}^2} +
\frac{2\zeta_{zi}}{\omega_{nz_i}}\omega j\right|
- \sum_i 20\log_{10}|1+j\omega\tau_{pi}| -\\
&-\sum_i 20 \log_{10}
\left|1-\frac{\omega^2}{\omega_{np_i}^2} +
\frac{2\zeta_{pi}}{\omega_{np_i}}\omega j\right|
\end{aligned}$$

Sono presenti quattro tipologie di termini, si possono tracciare singolarmente
e poi sommare tra di loro.
Facendo riferimento in questo caso ai poli si hanno i seguenti termini.
$$
W_a(j\omega) = K_B,\quad W_b(j\omega)=\frac{1}{j\omega},\quad W_c(j\omega)=
\frac{1}{1+j\omega\tau},\quad
W_d(j\omega)=\frac{1}{1-\frac{\omega^2}{\omega_n^2}+\frac{2\zeta\omega}{\omega_n
} }
$$

\textbf{Termine guadagno}
Il termine $W_a$ è un guadagno costante, non dipende dalla frequenza, viene
rappresentato come una retta orizzontale sul diagramma di Bode, maggiore o
minore di 0 se si ha un'amplificazione o un'attenuazione
\begin{figure}[h]
\centering
\begin{tikzpicture}[
gnuplot def/.append style={prefix={tikz/}},
yscale=2/20,xscale=7/5
]
\tikzset{
semilog lines/.style={black},
}
\UnitedB
\semilog{-2}{3}{-10}{10}
\BodeGraph{-2:3}{\KAmp{2}}
\BodeGraph{-2:3}{\KAmp{1}}
\BodeGraph{-2:3}{\KAmp{0.5}}
\end{tikzpicture}
\end{figure}

\textbf{Termine monomio}
Si rappresenta il termine $W_b$
$$
|W_b(j\omega)|_{\si{\deci\bel}} = 20\log_{10}\left|\frac{1}{j\omega}\right| =
-20\log_{10} |j\omega| = -20 \log_{10} (\omega)
$$
\begin{figure}[h]
\centering
\begin{tikzpicture}[
gnuplot def/.append style={prefix={tikz/}}]
\begin{scope}[xscale=7/4,yscale=3/80]
\tikzset{
semilog lines/.style={black},
}
\UnitedB
\semilog{-2}{2}{-40}{40}
\BodeGraph{-2:2}{\IntAmp{1}}
\BodeGraph{-1:1}{-20*2*log10(10**t)}
\end{scope}
\end{tikzpicture}
\caption{Guadagno polo nell'origine}
\label{fig.amplitude_origin_pole}
\end{figure}
Si ottiene una retta con pendenza \SI{-20}{\deci\bel/dec} (\textit{deciBel per
decade}), se ci fosse più di un polo nell'origine, si avrebbe un esponente $g$
sul termine $j\omega$ che diventerebbe un fattore moltiplicativo della pendenza
della retta, in questo caso se i poli sono due la retta ha una pendenza di
\SI{-40}{\deci\bel/dec}.
Viceversa se ci fossero degli zeri nell'origine, quindi $g$ negativo, si
avrebbe una pendenza della retta positiva.
Si usa spesso una pendenza compatta $-1,-2,+1,+2$ riferendosi al coefficiente
$g$ e non al prodotto completo per 20, quindi con una pendenza $-1$ si indica
\SI{-20}{\deci\bel/dec}, con $-2$ saranno \SI{-40}{\deci\bel/dec} e così via.

\newpage
\textbf{Termine binomio}
Si vuole rappresentare il modulo del binomio $W_c(j\omega)$
$$
|W_c(j\omega)| = 20\log_{10}\left|\frac{1}{1+j\omega\tau}\right| =
-20\log_{10}|1+j\omega \tau| = -20\log_{10}\sqrt{1+(\omega\tau)^2}
$$
Per studiare con comodità questa funzione può essere comodo eseguire prima
l'analisi asintotica, ossia l'analisi per $\omega$ molto piccoli e molto
grandi, si approssima la funzione
$$
|W_c(j\omega)| \simeq
\left\{\begin{aligned}
&-20\log_{10}\sqrt{1} = 0 &  & \omega \ll \frac{1}{|\tau|}\\
&-20\log_{10}|\omega\tau| &  & \omega \gg \frac{1}{|\tau|}
\end{aligned}\right.
$$
La retta in questo caso non passerà più per l'origine (1,0) ma per il punto
($\frac{1}{|\tau|}$,0), una decade dopo il punto di intersezione il diagramma
coincide con la retta asintotica e l'errore è trascurabile.
\begin{figure}[h]
\def\tau{0.5}
\centering
\begin{tikzpicture}[
gnuplot def/.append style={prefix={tikz/}}]
\begin{scope}[xscale=7/3,yscale=3/60]
\tikzset{
semilog lines/.style={black},
}
\UnitedB
\semilog{-1}{2}{-40}{20}
\BodeGraph[asymp lines,samples=100]{-1:2}
{\POAmpAsymp{1}{\tau}}
\BodeGraph[color=red]{-1:1.3}{-20*2*log10(sqrt(1 + (\tau*10**t)**2))}
\BodeGraph{-1:2}{\POAmp{1}{\tau}}
\end{scope}
\end{tikzpicture}
\caption{$\frac{1}{1+j\omega\cdot\tau},\quad
\textcolor{red}{\frac{1}{(1+j\omega\cdot\tau)^2}}$}
\label{fig.amplitude_binomio}
\end{figure}
È facile individuare il \textit{punto di rottura}, intersezione dei due
asintoti conoscendo $\tau$.
L'errore massimo si ha nel punto di rottura, si valuta la funzione in quel punto
$$
\left|W_c\left(j\frac{1}{|\tau|}\right)\right|_{\si{\deci\bel}} =
-20\log_{\si{\deci\bel}}\left|\frac{1}{1+j\frac{1}
{\cancel{|\tau|}}\cancel{\tau}
} \right| = -20\log_{10}\sqrt{2} \simeq \SI{-3}{\deci\bel}
$$
Il diagramma passa a \SI{-3}{\deci\bel} dal punto di rottura.
Se esistesse una molteplicità $k$ nel polo si avrebbe una pendenza della
curva pari a $-k$ superato il punto di rottura.
Viceversa se ci fosse uno zero anziché un polo si avrà una pendenza positiva.
Anche lo scostamento dal punto di rottura viene moltiplicato del coefficiente
$k$.

\newpage
\textbf{Fattore trinomio}
Si valuta il diagramma in ampiezza del fattore trinomio utilizzando gli asintoti
$$
|W_d(j\omega)|_{\si{\deci\bel}} = 20\log_{10} \left|\frac{1}
{1-\frac{\omega^2}{\omega_n^2}+j\frac{2\zeta\omega}{\omega_n}}\right|=
-20\log_{10}\sqrt{
\left(1-\frac{\omega^2}{\omega_n^2}\right)^2+\frac{4\zeta^2\omega^2}
{\omega_n^2} }
$$
Si confronta $\omega$ con $\omega_n$, $\zeta$ è minore di 1
$$
|W_d(j\omega)|_{\si{\deci\bel}} \simeq\left\{
\begin{aligned}
&-20\log_{10}\sqrt{1}\simeq 0 & &\omega\ll\omega_n \\
&-40\log_{10}\left(\frac{\omega}{\omega_n}\right) & & \omega\gg\omega^n
\end{aligned}\right.
$$
L'andamento è simile al termine binomio ma la pendenza è questa volta di
\SI{-40}{\deci\bel/dec}
\begin{figure}[h]
\centering
\def\Wn{10}
\def\zOne{0.2}
\def\K{1}
\def\zTwo{0.9}
\begin{tikzpicture}[
gnuplot def/.append style={prefix={tikz/}}]
\begin{scope}[xscale=7/3,yscale=3/60]
\tikzset{
semilog lines/.style={black},
}
\UnitedB
\semilog{-1}{2}{-40}{20}
\BodeGraph[asymp lines,samples=100]{-1:2}{\SOAmpAsymp{\K}{\zOne}{\Wn}}
\BodeGraph[color=red,thin,samples=150]{-1:2}{\SOAmp{\K}{\zOne}{\Wn}}
\BodeGraph{-1:2}{\SOAmp{\K}{\zTwo}{\Wn}}
\end{scope}
\end{tikzpicture}
\caption{$\textcolor{red}{\zeta=\zOne},\quad \textcolor{blue}{\zeta=\zTwo} $}
\label{fig.amplitude_trinomio}
\end{figure}

Se il termine di smorzamento $\zeta$ assume valori molto piccoli
($\zeta<\frac{1}{\sqrt{2}}\simeq0.707$) si ha un picco di massimo chiamato
\textit{picco di risonanza} e si presenta ad una frequenza di risonanza
$\omega_r$ più piccola del punto di rottura pari a
$$
\omega_r = \omega_n\sqrt{1-2\zeta^2}
$$
Il valore del picco di risonanza vale
$$
M_r = |W(j\omega_r)| = \frac{1}{2|\zeta|\sqrt{1-\zeta^2}}
$$
Il punto di passaggio nel punto di rottura vale
$$
|W(j\omega_n)| = \frac{1}{2|\zeta|}
$$
Il caso limite con $\zeta=0$ si ha un picco di risonanza infinito alla
pulsazione di risonanza $\omega_n$.
L'errore che si può commettere utilizzando le rette asintotiche va da un minimo
di \SI{-6}{\deci\bel} ad un massimo infinito, è necessario controllare il
valore di $\zeta$ e applicare le dovute correzioni.
Anche in questo caso la presenza di un coefficiente $k$ che rappresenti la
molteplicità dei poli, si avrà un'amplificazione dei picchi e della pendenza
dell'asintoto destro, nel caso in cui questo fosse negativo si avrebbe uno zero
e il ribaltamento del grafico, con una eventuale cancellazione di una frequenza
dall'uscita.

\newpage
\subsection{Regole di tracciamento}
Si elencano alcune regole per tracciare i diagrammi di Bode dei sistemi.
\begin{enumerate}
\item Scrivere la $W(j\omega)$ nella forma di Bode
\item Il tratto iniziale ha pendenza pari a $-g$ e alla frequenza $\omega=1$ (o
il suo prolungamento) assume valore $20\log_{10}|K_B| = [K_B]_{\si{\deci\bel}}$
\item In corrispondenza dei punti di rottura di zeri o poli reali la pendenza
aumenta o diminuisce, rispettivamente di un numero di unità pari alla
molteplicità dello zero o del polo corrispondente.
\item In corrispondenza dei punti di rottura di zeri o poli complessi la
pendenza
aumenta o diminuisce, rispettivamente di un numero \textit{doppio} di unità
pari alla
molteplicità dello zero o del polo corrispondente.
\item Se i punti di rottura sono distanti almeno di una decade è possibile
applicare le eventuali correzioni in quei punti, altrimenti se le frequenze di
rottura sono vicine tra loro, le correzioni si influenzerebbero.
\end{enumerate}

\newpage
\subsubsection{Esercizio di tracciamento}
\label{sec:Esercizio_bode}
Si consideri la seguente funzione
$$
W(s) = \frac{(s-1)(s+10)}{s(s^2+s+16)}
$$
Va posta nella forma di Bode
$$
W(j\omega) = \frac{(-1)(10)}{16} \cdot \frac{1}{{j\omega}}\cdot
\frac{(1-{j\omega})(1+0.1{j\omega})
}{\left( 1+\frac{1}{16}{j\omega} + \frac{{j\omega}^2}{16} \right)}
$$
$$
K_B = -\frac{10}{16} = -\frac{5}{8} \stackrel{\si{\deci\bel}}{\longrightarrow}
[K_B]_{\si{\deci\bel}} = -4,\ g=1,\ \tau_{z_1} = -1,\ \tau_{z_2} = 0.1
$$
Si ricavano inoltre la pulsazione naturale e la $\zeta$
$$
\omega_n = \sqrt{16} = 4,\quad \frac{2\zeta}{\omega_n} = \frac{1}{16}
\longrightarrow \zeta= \frac{1}{8} = 0.125 < \frac{1}{\sqrt{2}}
$$


I punti di rottura sono presenti in $1,\ 4,\ 10$ \si{\radian/\second},
si traccia il diagramma
\begin{figure}[h]
\centering
\begin{tikzpicture}[
gnuplot def/.append style={prefix={tikz/}}]
\begin{scope}[xscale=7/3,yscale=3/60]
\tikzset{
semilog lines/.style={black},
}
\UnitedB
\semilog{-1}{2}{-40}{20}
\BodeGraph[asymp
lines,samples=400]{-1:2}{\IntAmp{-0.625}-\POAmpAsymp{1}{0.1}-\POAmpAsymp{1}{1}
+\SOAmpAsymp{1}{0.125}{4}}
\BodeGraph[samples=400]{-1:2}{\IntAmp{-0.625}-\POAmp{1}{0.1}-\POAmp{1}{1}
 +\SOAmp{1}{0.125}{4}}
\end{scope}
\end{tikzpicture}
%\caption{$\textcolor{red}{\zeta=\zOne},\quad \textcolor{blue}{\zeta=\zTwo} $}
\end{figure}

Si tracciano inizialmente i diagrammi asintotici, il primo passa a quota
\SI{-4}{\deci\bel} con una pendenza di \SI{-20}{\deci\bel} (-2), poi si somma lo
zero centrato in 1, si somma 1 e si ha una pendenza nulla fino al punto di
rottura situato in $\omega=4$ con pendenza -2, (polo del secondo ordine),
fino al punto di rottura in $\omega=10$ in cui si somma nuovamente uno
zero, aumentando la pendenza di 1 quindi si termina asintoticamente la funzione
con una pendenza -1.
La pendenza finale è sempre pari all'ordine relativo $-(n-m)$.
