
\section{Diagrammi di Bode}
I \textit{diagrammi di Bode} sono strumenti grafici utili a rappresentare la
funzione di risposta armonica $W(j\omega)$ che permette di descrivere il
comportamento in frequenza dei sistemi dinamici.

I diagrammi di Bode si dividono in due grafici, uno per rappresentare il modulo
di $W(j\omega)$, l'altro per la fase e prendono rispettivamente il nome di
diagramma del modulo e diagramma di fase.

Spesso l'asse delle ascisse è rappresentato in scala logaritmica, data
l'elevata ampiezza del dominio delle funzioni, di conseguenza il rapporto tra
due pulsazioni equidistanziate è sempre pari a dieci.
Questa distanza viene chiamata \textit{decade}.

L'asse lineare viene invece rappresentato in scala lineare ma i valori
riportati sono espressi in \textit{deciBel}
$$
|W(j\omega)|_{\si{\deci\bel}} = 20\log_{10}|W(j\omega)|
$$
Si usa per comodità il fattore $20$ e non $10$ come da definizione.
Il grafico prende il nome di scala semilogaritmica.
\begin{figure}[h]
\centering
\begin{tikzpicture}[
yscale=2/30,xscale=7/5
]
\tikzset{
semilog lines/.style={black},
}
\semilog{-2}{3}{-20}{10}
\end{tikzpicture}
\end{figure}

Il diagramma della fase è anch'esso rappresentato in scala logaritmico e la
fase viene rappresentata in scala lineare, in gradi o radianti.

Uno strumento rapido per rappresentare i diagrammi di Bode è quello di
utilizzare i \textit{diagrammi asintotici}, una rappresentazione approssimata
ma molto rapida dei sistemi.
Conviene innanzitutto trasformare la funzione di trasferimento nella forma di
Bode
$$
W(j\omega) = \frac{K_B}{(j\omega)^g} \frac{\prod_i (1+j\omega \tau_{zi})
\prod_i (1-\frac{\omega^2}{\omega_{nz_i}^2} +
\frac{2\zeta_{pi}}{\omega_{np_i}}\omega j) }
{\prod_i (1+j\omega \tau_{pi})
\prod_i (1-\frac{\omega^2}{\omega_{np_i}^2} +
\frac{2\zeta_{pi}}{\omega_{np_i}}\omega j) }
$$

Si vuole tracciare il modulo in deciBel
$$
|W(j\omega)|_{\si{\deci\bel}} = 20\log_{10}|W(j\omega)|
$$
Sfruttando le proprietà del logaritmo
$$\begin{aligned}
|W(j\omega)|_{\si{\deci\bel}} &= 20\log_{10}|K_B|-20g\cdot\log_{10}|j\omega| +
\sum_i 20\log_{10}|1+j\omega\tau_{zi}| + \\
&+\sum_i 20 \log_{10}
\left|1-\frac{\omega^2}{\omega_{nz_i}^2} +
\frac{2\zeta_{zi}}{\omega_{nz_i}}\omega j\right|
- \sum_i 20\log_{10}|1+j\omega\tau_{pi}| -\\
&-\sum_i 20 \log_{10}
\left|1-\frac{\omega^2}{\omega_{np_i}^2} +
\frac{2\zeta_{pi}}{\omega_{np_i}}\omega j\right|
\end{aligned}$$

Sono presenti quattro tipologie di termini, si possono tracciare singolarmente
e poi sommare tra di loro.
Facendo riferimento in questo caso ai poli si hanno i seguenti termini.
$$
W_a(j\omega) = K_B,\quad W_b(j\omega)=\frac{1}{j\omega},\quad W_c(j\omega)=
\frac{1}{1+j\omega\tau},\quad
W_d(j\omega)=\frac{1}{1-\frac{\omega^2}{\omega_n^2}+\frac{2\zeta\omega}{\omega_n
} }
$$

\subsubsection{Termine guadagno}
Il termine $W_a$ è un guadagno costante, non dipende dalla frequenza, viene
rappresentato come una retta orizzontale sul diagramma di Bode, maggiore o
minore di 0 se si ha un'amplificazione o un'attenuazione
\begin{figure}[h]
\centering
\begin{tikzpicture}[
gnuplot def/.append style={prefix={tikz/}},
yscale=2/30,xscale=7/5
]
\tikzset{
semilog lines/.style={black},
}
\UnitedB
\semilog{-2}{3}{-20}{10}
\BodeGraph{-2:3}{\KAmp{2}}
\end{tikzpicture}
\end{figure}

\begin{figure}[h]
\centering
\begin{tikzpicture}[
gnuplot def/.append style={prefix={tikz/}},
yscale=2/30,xscale=7/5
]
\tikzset{
semilog lines/.style={black},
}
\UnitedB
\semilog{-2}{3}{-20}{10}
\BodeGraph{-2:3}{\KAmp{2}}
\end{tikzpicture}
\end{figure}
26:18
