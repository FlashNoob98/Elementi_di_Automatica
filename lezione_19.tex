
\section{Diagrammi di Bode}
I \textit{diagrammi di Bode} sono strumenti grafici utili a rappresentare la
funzione di risposta armonica $W(j\omega)$ che permette di descrivere il
comportamento in frequenza dei sistemi dinamici.

I diagrammi di Bode si dividono in due grafici, uno per rappresentare il modulo
di $W(j\omega)$, l'altro per la fase e prendono rispettivamente il nome di
diagramma del modulo e diagramma di fase.

Spesso l'asse delle ascisse è rappresentato in scala logaritmica, data
l'elevata ampiezza del dominio delle funzioni, di conseguenza il rapporto tra
due pulsazioni equidistanziate è sempre pari a dieci.
Questa distanza viene chiamata \textit{decade}.

L'asse lineare viene invece rappresentato in scala lineare ma i valori
riportati sono espressi in \textit{deciBel}
$$
|W(j\omega)|_{\si{\deci\bel}} = 20\log_{10}|W(j\omega)|
$$
Si usa per comodità il fattore $20$ e non $10$ come da definizione.
Il grafico prende il nome di scala semilogaritmica.
\begin{figure}[h]
\centering
\begin{tikzpicture}[
yscale=2/30,xscale=7/5
]
\tikzset{
semilog lines/.style={black},
}
\semilog{-2}{3}{-20}{10}
\end{tikzpicture}
\end{figure}

Il diagramma della fase è anch'esso rappresentato in scala logaritmico e la
fase viene rappresentata in scala lineare, in gradi o radianti.


