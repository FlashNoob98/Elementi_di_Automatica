\subsection{Teorema di Caley-Hamilton}
Si consideri una matrice $A$ quadrata
$$
A\in\mathbb{R}^{n\times n},\ p(\lambda) = \text{det}(\lambda I-A) = \lambda^n +
\alpha_{n-1}\lambda^{n-1} + \ldots + \alpha_0 \lambda^0
$$
Ogni matrice quadrata è soluzione del proprio polinomio caratteristico, con
abuso di notazione
$$
p(A) = A^n + \alpha_{n-1}A^{n-1}+ \ldots + \alpha_0 A^0 =
\underline{\underline{0}}
$$
con $A^0 = I$ e $p(A)$ un polinomio di matrici.

Si dimostra il teorema con l'ipotesi semplificativa ma non necessaria di
matrice A di semplice struttura
$$
p(A)\cdot u_i = A^nu_i + \alpha_{n-1}A^{n-1}u_i + \ldots + \alpha_0 u_i =
(\lambda_i^n+\alpha_{n-1}\lambda_i^{n-1}+\ldots + \alpha_0)u_i = \underline{0}\
\forall i
$$
Quello ottenuto all'ultimo termine è proprio il polinomio caratteristico che
valutato nell'autovalore è nullo per definizione.

Nell'ipotesi di semplice struttura i diversi autovettori $u_i$ formano una
base, di conseguenza tutti gli altri vettori saranno combinazione lineare degli
autovettori e restituiranno anch'essi il vettore nullo nulli se moltiplicati
per il polinomio di matrici $p(A)$
$$
p(A) = 0 \Rightarrow p(A)\cdot x = \underline{0} \ \ \forall x\in X^n
$$
La matrice deve necessariamente avere rango nullo perché moltiplicata per un
vettore dia come risultato un vettore nullo, di conseguenza, l'unica matrice di
rango nullo è la matrice nulla
$$
\rho(p(A))=0 \Rightarrow p(A) = \underline{\underline{0}}
$$

Si può isolare il termine $A^n$ dal polinomio di matrici, $k\geq 0$
$$
A^n = \sum_{i=0}^{n-1}(-\alpha_i)A^i \Rightarrow A^{n+k} =
\sum_{i=0}^{n-1}\beta_iA^i
$$
La potenza di ordine $n$ è una combinazione lineare di potenze fino all'ordine
$n-1$, si dimostra per induzione, vera per $k=0$, si ipotizza vera per un
generico $k$, si dimostra che è vera per $k+1$, si ottiene la dimostrazione del
teorema.

\subsection{Tecnica del polinomio interpolante - Matrice di Vandermonde}
Si vuole calcolare ancora la matrice esponenziale $e^{At}$
\begin{equation}
e^{At} \stackrel{\text{def}}{=} \sum_{k=0}^{+\infty} \frac{A^kt^k}{k!} =
\sum_{i=0}^{n-1} \alpha_i(t)A^i
\label{eq.:esponenziale_polinomio_interpolante}
\end{equation}
Sfruttando quanto ricavato dal teorema di Caley-Hamilton si potrebbe riscrivere
la potenza come una combinazione lineare di potenze di ordine inferiore.
I coefficienti $\alpha_i(t)$ raggruppano tutti i coefficienti della potenza
$i$-esima della matrice $A$, è presente la dipendenza del tempo a causa dei
termini $t^k$.
Se si potessero calcolare i coefficienti $\alpha_i(t)$ si potrebbe ridurre la
serie ad una somma finita di matrici.

$$
e^{At} \cdot u_1 = \sum_{i=0}^{n-1} \alpha_i(t)A^iu_1 =
\sum_{i=0}^{n-1}\alpha_i(t)\lambda^i_1 u_1 = e^{\lambda_1 t} u_1
$$
Si può ripetere questo passaggio per tutti gli autovettori della matrice $A$
$$
e^{At}u_n = \sum_{i=0}^{n-1} \alpha_i(t)A^iu_n = \sum_{i=0}^{n-1}
\alpha_i(t)\lambda_n^i u_n = e^{\lambda_n t} u_n
$$
Essendo gli autovettori non nulli allora
$$
\sum_{i=0}^{n-1} \alpha_i(t) \lambda^i_n = e^{\lambda_n t}
$$
Riscrivendo il sistema in forma compatta
$$
\begin{bmatrix}
1 &\lambda_1 & \lambda_1^2 & \ldots & \lambda_1^{n-1}\\
\vdots &\vdots  &\vdots    &              & \vdots&         \\
1  &\lambda_n & \lambda_n^2 & \ldots &\lambda_n^{n-1}
\end{bmatrix}
\begin{bmatrix}
                \alpha_0(t) \\ \vdots \\ \alpha_{n-1}(t)
                \end{bmatrix} = \begin{bmatrix}
                                e^{\lambda_1 t} \\ \vdots \\ e^{\lambda_n t}
                                \end{bmatrix}
$$
La prima matrice prende il nome di \textbf{matrice di Vandermonde}, si indica
con \linebreak $V(\lambda_1,\ldots,\lambda_n)$, funzione di $n$ elementi.

Il determinante della matrice di Van der Monde è pari alla produttoria delle
differenze di tutti i coefficienti
$$
\text{det}\left(V\left(\lambda_1,\ldots,\lambda_n\right)\right) = \prod_{1\leq
i < j \leq n } (\lambda_j - \lambda_i)
$$
L'unico caso che può annullare la produttoria è quando due coefficienti con
indice diverso siano uguali, ovvero $A$ non sia di semplice struttura, in caso
contrario, invertendo la matrice si possono ricavare i coefficienti
$$
\begin{bmatrix}
\alpha_0(t)\\
\vdots \\
\alpha_{n-1}(t)
\end{bmatrix} =
V^{-1}(\lambda_1,\ldots,\lambda_n)
\begin{bmatrix}
e^{\lambda_1 t}\\
\vdots \\
e^{\lambda_n t}
\end{bmatrix}
$$
Sostituiti nell'equazione \ref{eq.:esponenziale_polinomio_interpolante}, dopo
aver calcolato le potenze della matrice $A$ permettono di calcolare la matrice
esponenziale con la somma di finiti termini.

Rispetto al metodo precedente non è necessario calcolare gli autovettori ma
\textbf{è necessario che la matrice $A$ sia di semplice struttura}.

\subsubsection{Procedura operativa}
\begin{enumerate}
 \item Calcolo degli autovalori
 \item Verifica che i $\lambda_i$ siano distinti, ossia che $A$ sia semplice
 \item Se si la 2. risoluzione del sistema
$V(\lambda_1,\ldots,\lambda_n)\begin{bmatrix}
\alpha_0(t) \\ \vdots \\ \alpha_{n-1}(t) \end{bmatrix}= \begin{bmatrix}
e^{\lambda_1 t } \\ \vdots \\ e^{\lambda_n t}
\end{bmatrix}$
\item Calcolo della matrice esponenziale $e^{At} = \sum_{i=0}^{n-1}
\alpha_i(t)A^i$
\end{enumerate}

\subsubsection{Esempio numerico}
Si riprende la stessa matrice utilizzata nel metodo della diagonalizzazione
(vedi paragrafo \ref{sec.:metodo_diagonalizzazione})
$$
A = \begin{bmatrix}
0 & 1\\
-2 & -3
\end{bmatrix}
$$
\begin{enumerate}
\item $p(\lambda) = \text{det} (\lambda I -A) = 0\Rightarrow \lambda_1 = -1,\
\lambda_2 = -2$
\item $\lambda_1\neq \lambda_2 \Rightarrow A $ semplice struttura
\item Costruzione e risoluzione del sistema $V\alpha_n(t) = e^{\lambda_n t}$
$$
\begin{bmatrix}
1 & -1 \\
1 & -2
\end{bmatrix}\begin{bmatrix}
\alpha_0(t) \\ \alpha_1 (t)
\end{bmatrix} =
\begin{bmatrix}
e^{-t} \\
e^{-2t}
\end{bmatrix}\Rightarrow \begin{bmatrix}
\alpha_0(t) \\ \alpha_1 (t)
\end{bmatrix} = - \begin{bmatrix}
-2 & 1 \\
-1 & 1
\end{bmatrix}\begin{bmatrix}
e^{-t} \\ e^{-2t}
\end{bmatrix}
$$
Sviluppando si ottiene
$$\left\{\begin{aligned}
\alpha_0(t) &= 2e^{-t} - e^{-2t}\\
\alpha_1(t) &= e^{-t} -e^{-2t}
\end{aligned}\right.$$
\item Calcolo della matrice esponenziale
$$\begin{aligned}
e^{At} &= \alpha_0(t)I + \alpha_1(t) A^1 =\\
&=\begin{pmatrix}
2e^{-t}-e^{-2t} & 0 \\
0 & 2e^{-t}-e^{-2t}
\end{pmatrix} + \begin{pmatrix}
0 & e^{-t}-e^{-2t} \\
-2e^{-t}+2e^{-2t} & -3e^{-t} + 3e^{-2t}
\end{pmatrix} =\\
&= \begin{pmatrix}
2e^{-t}-e^{-2t} & e^{-t}-e^{-2t}\\
-2e^{-t} + e^{-2t} & -e^{-t}+2e^{2t}
\end{pmatrix}
\end{aligned}$$
\end{enumerate}

\newpage
\subsubsection{Riassunto dei due metodi precedenti}
Sia $A$ una matrice di semplice struttura, si può calcolare la matrice
esponenziale mediante due metodi
\begin{enumerate}
 \item Diagonalizzazione

 Per $n$ autovalori distinti
 $$
 \exists\ n \text{ autovettori indipendenti } u_i \Rightarrow U=[u_1,\ldots,u_n]
 $$
 $$\begin{aligned}
 A &= U\Lambda U^{-1} = \sum_{i=1}^{n} \lambda_i u_iv_i^T\\
 e^{At} &= Ue^{\Lambda t} U^{-1} = \sum_{i=1}^{n} e^{\lambda_i}u_iv_i^T
 \end{aligned}$$
 Ricordando che il prodotto tra autovettori e le colonne della matrice modale
inversa si ottengono le matrici $R_i$ denominate \textit{matrici dei residui
polari}, avranno rango unitario in questo caso, hanno in generale il rango pari
alla molteplicità geometrica dell'autovettore $i-$esimo $u_i$.
$$
u_i v_i^T = R_i
$$

\item Polinomio interpolante

Si costruisce la matrice di Vandermonde, si calcola il vettore dei coefficienti
$\alpha$ e si calcola la matrice esponenziale
$$
V(\lambda_1,\ldots,\lambda_n) \Rightarrow \alpha_i(t):
\begin{bmatrix}
\alpha_0(t) \\
\vdots \\
\alpha_{n-1}(t)
\end{bmatrix} = V^{-1}(\lambda_1,\ldots,\lambda_n)
\begin{bmatrix}
e^{\lambda_1t}\\
\vdots \\
e^{\lambda_nt}
\end{bmatrix}
$$
$$
e^{At} =\sum_{i=0}^{n-1} \alpha_i(t)A^i
$$
\end{enumerate}

\subsection{Matrice non semplice}
Si suppone che la matrice non sia di semplice struttura ma solo
diagonalizzabile, esisteranno $m$ autovalori distinti $(m < n)$, dunque il
generico autovalore $\lambda_i$ avrà molteplicità algebrica maggiore di uno.
Per calcolare gli autovettori si risolvono i sistemi
$$
(A-\lambda_iI)u_i=0
$$
la molteplicità geometrica del generico autovettore è
$$
m_{gi}= n-\rho(A-\lambda_iI)
$$
Se la molteplicità geometrica è inferiore di quella algebrica, si troveranno
meno autovettori di quanti sono necessari a diagonalizzare la matrice che
risulterà dunque \textbf{non diagonalizzabile}. $1< m_{gi} \leq  m_{ai} \leq n $

Considerato il generico autovalore $\lambda_i$ si pongono i corrispondenti
autovettori in colonna
$$
\lambda_i \rightarrow U_i=[u_{i_1},\ldots,u_{i_{m_{ai}}}]
$$
Si costruisce in seguito la matrice modale affiancando i blocchi appena ottenuti
$$
U = [U_1,U_2,\ldots,U_m]
$$
Di conseguenza si costruisce una matrice degli autovalori dove ogni autovalore
è ripetuto sulla diagonale per un numero di volte pari alla sua molteplicità
(algebrica o geometrica, coincidono)
$$
A=U
\begin{bmatrix}
\lambda_1 &0 &0 &0 &0 &0 &0 &0 &0 &0 \\
0    &\ddots &0 &0 &0 &0 &0 &0 &0 &0 \\
0 &0 &\lambda_1 &0 &0 &0 &0 &0 &0 &0 \\
0 &0 &0 &\lambda_2 &0 &0 &0 &0 &0 &0 \\
0 &0 &0 &0    &\ddots &0 &0 &0 &0 &0 \\
0 &0 &0 &0 &0 &\lambda_2 &0 &0 &0 &0 \\
0 &0 &0 &0 &0 &0    &\ddots &0 &0 &0 \\
0 &0 &0 &0 &0 &0 &0 &\lambda_n &0 &0 \\
0 &0 &0 &0 &0 &0 &0 &0    &\ddots &0 \\
0 &0 &0 &0 &0 &0 &0 &0 &0 &\lambda_n
\end{bmatrix}U^{-1}
$$
Raccogliendo i termini
$$
A = \sum_{i=1}^m \lambda_i \cdot \sum_{j=1}^{m_{ai}} u_{ij}\cdot v_{ij}^T
$$
Sono necessari due pedici in quanto il primo indica il gruppo di autovettori
associati all'autovalore $i$ mentre il secondo è associato alla molteplicità
$m_{ai}$ e permette di moltiplicare l'autovalore per ogni suo autovettore.

Analogamente per la matrice esponenziale
$$
e^{At} = \sum_{i=1}^m e^{\lambda_i t} \cdot \sum_{j=1}^{m_{ai}} u_{ij}\cdot
v_{ij}^T
$$
Con un abuso di notazione si potrebbero indicare autovalori uguali con nomi
diversi, in modo da avere comunque un numero di autovalori pari al numero di
autovettori e riprendere la notazione del caso precedente con matrice semplice.

I vettori  $u_i$ vengono chiamati autovettori \textit{destri} ovvero devono
moltiplicare a destra la matrice $A$ per restituire l'autovalore.

Gli autovettori $v_i$ vengono invece chiamati autovettori sinistri (presi
sempre per colonna)
$$\begin{aligned}
u_i &: Au_i = \lambda_i u_i\\
v_i &: v_iA = \lambda_i v_i
\end{aligned}$$

\subsection{Autovalori complessi e coniugati}
Ad ogni coppia di autovalori complessi e coniugati corrisponderà una coppia di
autovettori complessi e coniugati, il resto della notazione non cambia, si
risolveranno due sistemi di equazioni, uno per la parte reale ed uno per la
parte immaginaria.

Si introduce una variante della diagonalizzazione, chiamata \textbf{forma
reale} della matrice degli autovalori.
Per avere tutti termini reali sulla matrice degli autovalori sarà necessario
perdere l'ipotesi di matrice diagonale ma di avere invece una matrice diagonale
a blocchi.

Si considera per semplicità di analisi un caso di $n=3$ ed un autovalore reale
ed uno complesso e coniugato, in caso contrario si ripete il risultato ottenuto.

I tre autovalori saranno
$$\left\{\begin{matrix}
\lambda \\ \alpha+j\omega \\ \alpha-j\omega
\end{matrix}\right.$$
Si devono costruire gli autovettori complessi e coniugati
$$\left\{\begin{aligned}
&(A-(\alpha+j\omega)I)(u_a+ju_b) = \underline{0} + j\underline{0}\\
&(A-(\alpha-j\omega)I)(u_a-ju_b) = \underline{0} + j\underline{0}
\end{aligned}\right.$$
Si sviluppa la prima riga
$$\begin{aligned}
&(A-(\alpha+j\omega))(u_a + ju_b) = \underline{0} +
j\underline{0}\Rightarrow
\left\{\begin{aligned}
&Au_a - \alpha u_a + \omega u_b = 0\\
&Au_b - \omega u_a -\alpha u_b = 0
\end{aligned}\right.
\end{aligned}$$
Si porta a sinistra la matrice $A$ e si lasciano gli altri termini a destra
$$
\left\{\begin{aligned}
Au_a &= \alpha u_a - \omega u_b \\
Au_b &= \omega u_a + \alpha u_b
\end{aligned}\right. \Leftrightarrow A (u_a \ u_b) = (u_a \ u_b)
\begin{pmatrix}
\alpha & \omega\\
-\omega & \alpha
\end{pmatrix}
$$

Se si volesse diagonalizzare, si dovrebbe costruire una matrice modale complessa
$$
U=\begin{pmatrix}u & u_a+ju_b & u_a-ju_b\end{pmatrix}
\stackrel{\Lambda=U^{-1}AU}{\rightarrow}\Lambda=
\begin{pmatrix}
\lambda & 0 & 0\\
0 & \alpha+j\omega & 0 \\
0 & 0 & \alpha -j\omega
\end{pmatrix}
$$
Quindi la matrice esponenziale
$$\begin{aligned}
e^{\Lambda t} &= \begin{pmatrix}
e^{\lambda t} & 0 & 0 \\
0 & e^{(\alpha +j\omega)t} & 0 \\
0 & 0 & e^{(\alpha - j\omega)t}
\end{pmatrix} =\\
&=\begin{pmatrix}
e^{\lambda t} & 0 & 0 \\
0 & e^{\alpha t}(\cos(\omega t) + j\sin(\omega t)) & 0\\
0 & 0 & e^{\alpha t}(\cos(\omega t) - j\sin(\omega t))
\end{pmatrix}
\end{aligned}$$
Questa matrice va poi calcolata per le matrici $U$ e $U^{-1}$ con termini
complessi.

Si sfrutta ora la matrice $U_r$ reale, considerando solo termini reali
$$
U_r = \begin{pmatrix}
u & u_a & u_b
\end{pmatrix}
$$
La matrice $\Lambda$ non sarà più diagonale
$$
AU_r = A\begin{pmatrix}
u & u_a & u_b
\end{pmatrix} =
\begin{pmatrix}
u & u_a & u_b
\end{pmatrix}\left(
\begin{array}{c:cc}
\lambda & 0 & 0\\ \hdashline
0 &\alpha &\omega \\
0 &-\omega &\alpha \\
\end{array}\right)
$$
Si è ottenuta una matrice diagonale a blocchi in cui il primo blocco è scalare
mentre alla coppia di autovalori complessi coniugati corrisponde il blocco
$2\times2$
$$
AU_r = U_r \Lambda_r
$$

Si calcola la matrice esponenziale
$$
e^{At} = U_r e^{\Lambda_r t}U_r^{-1}
$$
Si deve calcolare la matrice $e^{\Lambda_r t}$, sapendo che la matrice potenza
di una matrice diagonale a blocchi è ancora una matrice diagonale a blocchi.

$$
e^{At} = U_r \left(
\begin{array}{c:c}
e^{\lambda t} & 0 \quad 0 \\ \hdashline
\begin{aligned}
0 \\ 0
\end{aligned} &\begin{aligned} \\ e^{\begin{pmatrix}
\alpha& \omega\\
-\omega & \alpha
\end{pmatrix}t}\end{aligned}
\end{array}
\right) U_r^{-1}
$$
il termine esponenziale è ancora una matrice $2\times 2$
$$
e^{\begin{pmatrix}
\alpha& \omega\\
-\omega & \alpha
\end{pmatrix}t} =
e^{\alpha t} \begin{pmatrix}
\cos \omega t & \sin \omega t \\
-\sin \omega t & \cos \omega t
\end{pmatrix}
$$
per dimostrare quest'uguaglianza si devono verificare due condizioni:
\begin{enumerate}
 \item $e^{\Lambda_r \cdot 0} = I$
 \item $\frac{d}{dt} e^{\Lambda_r t} = \Lambda_re^{\Lambda_r t}$
\end{enumerate}
similmente alle condizioni dell'esponenziale scalare, entrambe si dimostrano
banalmente.

\newpage
In conclusione la matrice esponenziale
$$\begin{aligned}
e^{At} &= U_r \left(\begin{array}{c:cc}
e^{\lambda t} & 0 & 0 \\ \hdashline
0 & e^{\alpha t}\cos\omega t & e^{\alpha t}\sin\omega t \\
0 & -e^{\alpha t}\sin\omega t & e^{\alpha t}\cos\omega t
\end{array}\right)U_r^{-1} = \\
&=e^{\alpha t}uv^T + e^{\alpha t}\left[
u_a(\cos\omega t\cdot v_a^T + \sin\omega t\cdot v_b^T) +
u_b(\sin\omega t\cdot v_b^T - \cos\omega t\cdot v_a^T)
\right]
\end{aligned}$$
con la matrice inversa pari a
$$
U_r^{-1} = \begin{pmatrix}
v^T \\ v_a^T \\ v_b^T
\end{pmatrix}
$$
