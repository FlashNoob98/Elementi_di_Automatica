\subsection{Teorema di Caley-Hamilton}
Si consideri una matrice $A$ quadrata
$$
A\in\mathbb{R}^{n\times n},\ p(\lambda) = \text{det}(\lambda I-A) = \lambda^n +
\alpha_{n-1}\lambda^{n-1} + \ldots + \alpha_0 \lambda^0
$$
Ogni matrice quadrata è soluzione del proprio polinomio caratteristico, con
abuso di notazione
$$
p(A) = A^n + \alpha_{n-1}A^{n-1}+ \ldots + \alpha_0 A^0 =
\underline{\underline{0}}
$$
con $A^0 = I$ e $p(A)$ un polinomio di matrici.

Si dimostra il teorema con l'ipotesi semplificativa ma non necessaria di
matrice A di semplice struttura
$$
p(A)\cdot u_i = A^nu_i + \alpha_{n-1}A^{n-1}u_i + \ldots + \alpha_0 u_i =
(\lambda_i^n+\alpha_{n-1}\lambda_i^{n-1}+\ldots + \alpha_0)u_i = \underline{0}\
\forall i
$$
Quello ottenuto all'ultimo termine è proprio il polinomio caratteristico che
valutato nell'autovalore è nullo per definizione.

Nell'ipotesi di semplice struttura i diversi autovettori $u_i$ formano una
base, di conseguenza tutti gli altri vettori saranno combinazione lineare degli
autovettori e restituiranno anch'essi il vettore nullo nulli se moltiplicati
per il polinomio di matrici $p(A)$
$$
p(A) = 0 \Rightarrow p(A)\cdot x = \underline{0} \ \ \forall x\in X^n
$$
La matrice deve necessariamente avere rango nullo perché moltiplicata per un
vettore dia come risultato un vettore nullo, di conseguenza, l'unica matrice di
rango nullo è la matrice nulla
$$
\rho(p(A))=0 \Rightarrow p(A) = \underline{\underline{0}}
$$

Si può isolare il termine $A^n$ dal polinomio di matrici, $k\geq 0$
$$
A^n = \sum_{i=0}^{n-1}(-\alpha_i)A^i \Rightarrow A^{n+k} =
\sum_{i=0}^{n-1}\beta_iA^i
$$
La potenza di ordine $n$ è una combinazione lineare di potenze fino all'ordine
$n-1$, si dimostra per induzione, vera per $k=0$, si ipotizza vera per un
generico $k$, si dimostra che è vera per $k+1$, si ottiene la dimostrazione del
teorema.

\subsection{Tecnica del polinomio interpolante - Matrice di Vandermonde}
Si vuole calcolare ancora la matrice esponenziale $e^{At}$
\begin{equation}
e^{At} \stackrel{\text{def}}{=} \sum_{k=0}^{+\infty} \frac{A^kt^k}{k!} =
\sum_{i=0}^{n-1} \alpha_i(t)A^i
\label{eq.:esponenziale_polinomio_interpolante}
\end{equation}
Sfruttando quanto ricavato dal teorema di Caley-Hamilton si potrebbe riscrivere
la potenza come una combinazione lineare di potenze di ordine inferiore.
I coefficienti $\alpha_i(t)$ raggruppano tutti i coefficienti della potenza
$i$-esima della matrice $A$, è presente la dipendenza del tempo a causa dei
termini $t^k$.
Se si potessero calcolare i coefficienti $\alpha_i(t)$ si potrebbe ridurre la
serie ad una somma finita di matrici.

$$
e^{At} \cdot u_1 = \sum_{i=0}^{n-1} \alpha_i(t)A^iu_1 =
\sum_{i=0}^{n-1}\alpha_i(t)\lambda^i_1 u_1 = e^{\lambda_1 t} u_1
$$
Si può ripetere questo passaggio per tutti gli autovettori della matrice $A$
$$
e^{At}u_n = \sum_{i=0}^{n-1} \alpha_i(t)A^iu_n = \sum_{i=0}^{n-1}
\alpha_i(t)\lambda_n^i u_n = e^{\lambda_n t} u_n
$$
Essendo gli autovettori non nulli allora
$$
\sum_{i=0}^{n-1} \alpha_i(t) \lambda^i_n = e^{\lambda_n t}
$$
Riscrivendo il sistema in forma compatta
$$
\begin{bmatrix}
1 &\lambda_1 & \lambda_1^2 & \ldots & \lambda_1^{n-1}\\
\vdots &\vdots  &\vdots    &              & \vdots&         \\
1  &\lambda_n & \lambda_n^2 & \ldots &\lambda_n^{n-1}
\end{bmatrix}
\begin{bmatrix}
                \alpha_0(t) \\ \vdots \\ \alpha_{n-1}(t)
                \end{bmatrix} = \begin{bmatrix}
                                e^{\lambda_1 t} \\ \vdots \\ e^{\lambda_n t}
                                \end{bmatrix}
$$
La prima matrice prende il nome di \textbf{matrice di Vandermonde}, si indica
con \linebreak $V(\lambda_1,\ldots,\lambda_n)$, funzione di $n$ elementi.

Il determinante della matrice di Van der Monde è pari alla produttoria delle
differenze di tutti i coefficienti
$$
\text{det}\left(V\left(\lambda_1,\ldots,\lambda_n\right)\right) = \prod_{1\leq
i < j \leq n } (\lambda_j - \lambda_i)
$$
L'unico caso che può annullare la produttoria è quando due coefficienti con
indice diverso siano uguali, ovvero $A$ non sia di semplice struttura, in caso
contrario, invertendo la matrice si possono ricavare i coefficienti
$$
\begin{bmatrix}
\alpha_0(t)\\
\vdots \\
\alpha_{n-1}(t)
\end{bmatrix} =
V^{-1}(\lambda_1,\ldots,\lambda_n)
\begin{bmatrix}
e^{\lambda_1 t}\\
\vdots \\
e^{\lambda_n t}
\end{bmatrix}
$$
Sostituiti nell'equazione \ref{eq.:esponenziale_polinomio_interpolante}, dopo
aver calcolato le potenze della matrice $A$ permettono di calcolare la matrice
esponenziale con la somma di finiti termini.

Rispetto al metodo precedente non è necessario calcolare gli autovettori ma
\textbf{è necessario che la matrice $A$ sia di semplice struttura}.

\subsubsection{Procedura operativa}
\begin{enumerate}
 \item Calcolo degli autovalori
 \item Verifica che i $\lambda_i$ siano distinti, ossia che $A$ sia semplice
 \item Se si la 2. risoluzione del sistema
$V(\lambda_1,\ldots,\lambda_n)\begin{bmatrix}
\alpha_0(t) \\ \vdots \\ \alpha_{n-1}(t) \end{bmatrix}= \begin{bmatrix}
e^{\lambda_1 t } \\ \vdots \\ e^{\lambda_n t}
\end{bmatrix}$
\item Calcolo della matrice esponenziale $e^{At} = \sum_{i=0}^{n-1}
\alpha_i(t)A^i$
\end{enumerate}

\subsubsection{Esempio numerico}
Si riprende la stessa matrice utilizzata nel metodo della diagonalizzazione
(vedi paragrafo \ref{sec.:metodo_diagonalizzazione})
$$
A = \begin{bmatrix}
0 & 1\\
-2 & -3
\end{bmatrix}
$$
\begin{enumerate}
\item $p(\lambda) = \text{det} (\lambda I -A) = 0\Rightarrow \lambda_1 = -1,\
\lambda_2 = -2$
\item $\lambda_1\neq \lambda_2 \Rightarrow A $ semplice struttura
\item Costruzione e risoluzione del sistema $V\alpha_n(t) = e^{\lambda_n t}$
$$
\begin{bmatrix}
1 & -1 \\
1 & -2
\end{bmatrix}\begin{bmatrix}
\alpha_0(t) \\ \alpha_1 (t)
\end{bmatrix} =
\begin{bmatrix}
e^{-t} \\
e^{-2t}
\end{bmatrix}\Rightarrow \begin{bmatrix}
\alpha_0(t) \\ \alpha_1 (t)
\end{bmatrix} = - \begin{bmatrix}
-2 & 1 \\
-1 & 1
\end{bmatrix}\begin{bmatrix}
e^{-t} \\ e^{-2t}
\end{bmatrix}
$$
Sviluppando si ottiene
$$\left\{\begin{aligned}
\alpha_0(t) &= 2e^{-t} - e^{-2t}\\
\alpha_1(t) &= e^{-t} -e^{-2t}
\end{aligned}\right.$$
\item Calcolo della matrice esponenziale
$$\begin{aligned}
e^{At} &= \alpha_0(t)I + \alpha_1(t) A^1 =\\
&=\begin{pmatrix}
2e^{-t}-e^{-2t} & 0 \\
0 & 2e^{-t}-e^{-2t}
\end{pmatrix} + \begin{pmatrix}
0 & e^{-t}-e^{-2t} \\
-2e^{-t}+2e^{-2t} & -3e^{-t} + 3e^{-2t}
\end{pmatrix} =\\
&= \begin{pmatrix}
2e^{-t}-e^{-2t} & e^{-t}-e^{-2t}\\
-2e^{-t} + e^{-2t} & -e^{-t}+2e^{2t}
\end{pmatrix}
\end{aligned}$$
\end{enumerate}
51:10
