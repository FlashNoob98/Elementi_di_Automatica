
\subsection{Esempio Massa-Molla-Smorzatore con due masse}
Si vuole analizzare il modello ISU del seguente sistema composto da due masse
disposte nel seguente modo:
\begin{figure}[h]
 \centering
 \includegraphics[width=\picwid]{massa_molla_smorzatore_esempio_2.png}
 \label{fig:massa_molla_smorzatore_esempio_2}
\end{figure}

Si individua il sistema di riferimento e si indica con $s_1$ la posizione della
prima massa ed $s_2$ la seconda.
Esiste un'unica direzione di movimento e sono presenti due masse, dunque si
scriveranno due equazioni.
$$\left\{\begin{aligned}
m_1\ddot{s}_1 &= k_2(s_2-s_1) + b_2(\dot{s}_2-\dot{s}_1) -k_1s_1 - b_1\dot{s}_1
\\
m_2\ddot{s}_2 &= f - k_2(s_2-s_1) - b_2(\dot{s}_2-\dot{s}_1)
\end{aligned}\right.$$

Si analizzano le variabili del sistema
$$\begin{matrix}
f & s_1& \dot{s}_1 & s_2 &\dot{s}_2 \\
(0) & (2) & & (2) \\
u & x_1 & x_2 & x_3 & x_4
\end{matrix}$$

L'ordine del sistema è dunque quattro $(n=4)$, dato che ogni variabile non di
ingresso viene differenziata due volte.

Si riscrivono le equazioni in forma ISU
$$\left\{\begin{aligned}
\dot{x}_1 &= \dot{s}_1 = x_2 \\
\dot{x}_2 &= \ddot{s}_1 = \frac{k_2}{m_1}(x_3-x_1) + \frac{b_2}{m_1} (x_4-x_2)
- \frac{k_1}{m_1}x_1 - \frac{b_1}{m_1}x_2\\
\dot{x}_3 &=\dot{s}_2 = x_4 \\
\dot{x}_4 &= \ddot{s}_2 = \frac{1}{m_2}u -\frac{k_2}{m_2}(x_3-x_1) -
\frac{b_2}{m_2}(x_4 - x_2)\\
\ &\text{Uscite assegnate}\\
y_1 &= s_1 = x_1 \\
y_2 &= s_2 = x_3
\end{aligned}\right.$$

\newpage
Il sistema è lineare tempo invariante strettamente causale, si può porre il
sistema in forma matriciale compatta
$$x =
\begin{pmatrix}
 x_1 \\ x_2 \\ x_3 \\ x_4
\end{pmatrix} \quad
y= \begin{pmatrix}
    y_1 \\ y_2
   \end{pmatrix}
$$
$$\text{ISU} = \left\{\begin{aligned}
\dot{x} &= \begin{pmatrix}
0 & 1 & 0 & 0 \\
\\
-\frac{k_1+k_2}{m_1} & -\frac{b_1+b_2}{m_1} &
\frac{k_2}{m_1} & \frac{b_2}{m_1}\\
\\
0 & 0 & 0 & 1\\ \\
\frac{k_2}{m_2} & \frac{b_2}{m_2} & -\frac{k_2}{m_2} & -\frac{b_2}{m_2}
          \end{pmatrix}x +
          \begin{pmatrix}
        0 \\ 0 \\ 0 \\ \frac{1}{m}
          \end{pmatrix}u\\
y &=      \begin{pmatrix}
           1 & 0 & 0 & 0 \\
           0 & 0 & 1 & 0
          \end{pmatrix}x
\end{aligned}\right.$$

\subsection{Moto rotazionale}
Si considera un primo asse di movimento caratterizzato da una certa inerzia
$J_1$, connesso ad un secondo asse di momento $J_2$ mediante un elemento
elastico di costante $k$.

Si suppone che il primo asse subisca un effetto di attrito viscoso mediante un
coefficiente $b_1$ mentre il secondo subisce un attrito con coefficiente $b_2$.

Si suppone di applicare una coppia \textit{motrice} $\tau_m$ al primo asse ed
una coppia resistente $\tau_r$ in verso opposto al secondo asse.

Questo schema può rappresentare un esempio di accoppiamento non perfettamente
rigido tra l'asse di un motore ed un montacarichi (o un ascensore).
Si suppone il verso positivo degli spostamenti quello concorde con la coppia
$\tau_1$.

Si analizzano le equazioni, ne serviranno due, data la presenza di due inerzie.

35:07
