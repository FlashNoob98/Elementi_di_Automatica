
\subsection{Diagonalizzazione per similitudine}
Sia un'applicazione $A\in\mathbb{R}^{n\times n}$ di semplice struttura, ovvero
con $n$ autovalori distinti e di conseguenza $n$ autovettori distinti e
indipendenti tra loro, in uno spazio di $n$ dimensioni, dunque forniscono una
base.
$$
\left\{\begin{aligned}
\lambda_1& \rightarrow u_1\\
\vdots& \quad\ \  \vdots \\
\lambda_n & \rightarrow u_n
\end{aligned}\right.
$$
Si pongono i vettori per colonna in una matrice quadrata $U$
$$
U = [u_1, \ldots, u_n]
$$
Il rango della matrice sarà certamente $n$, composta da $n$ vettori linearmente
indipendenti, sarà una matrice non singolare.
La matrice $U$ può essere usata per realizzare una trasformazione di
similitudine.
$$
x = U\tilde{x}
$$
La matrice $U$ prende il nome di \textit{matrice modale},
la matrice della dinamica del sistema si può riscrivere come
$$
\tilde{A} = U^{-1}AU
$$
con $\tilde{A}$ la matrice di similitudine dello stesso sistema in una base
differente, \textit{equivalente} alla matrice $A$ di partenza.

Sviluppando i prodotti
$$
\tilde{A} = U^{-1}AU = \begin{bmatrix}
                        \lambda_1 & 0 & 0 \\
                        0 & \ddots & 0 \\
                        0 & 0 & \lambda_n
                        \end{bmatrix} = \Lambda
$$
La matrice $\Lambda$ viene denominata \textit{matrice degli autovalori}.
Si è ottenuta una matrice della dinamica del sistema di tipo diagonale, con
solo gli autovalori sulla diagonale.
La trasformazione di similitudine non altera gli autovalori del sistema.

Questa rappresentazione prende il nome di \textit{diagonalizzazione per
similitudine}, si è diagonalizzata la matrice della dinamica.

Si vuole dimostrare la precedente
$$\begin{aligned}
U^{-1} A[u_1 \ldots u_n] &= U^{-1} [Au_1, Au_2, \ldots, Au_n] =
U^{-1}[\lambda_1u_1,\lambda_2u_2,\ldots,\lambda_nu_n] =\\
&=U^{-1} [u_1,u_2,\ldots,u_n] \begin{bmatrix}
                        \lambda_1 & 0 & 0 \\
                        0 & \ddots & 0 \\
                        0 & 0 & \lambda_n
                             \end{bmatrix} = I\Lambda = \Lambda
\end{aligned}$$

\chapter{Analisi dei sistemi LTI nel dominio del tempo}
La forma ISU dei sistemi di questo tipo è
$$\left\{\begin{aligned}
\dot x &= Ax +Bu \\
y &= Cx + Du
\end{aligned}\right.$$
Le matrici caratteristiche sono costanti, ma il legame rispetto allo stato è
implicito, per ricavare esplicitamente il legame tra uscita e stato è necessario
integrare il sistema.

\section{Risoluzione analitica di un sistema differenziale}
Per risolvere il sistema è necessario determinare l'integrale generale, dato
dalla somma dell'integrale dell'omogenea associata
nominata $x_h(t)$ e di una qualunque soluzione particolare $x_p(t)$
$$
x(t) =x_h(t) + x_p(t)
$$
Soluzione omogenea $\dot x_h = Ax_h$, si suppone che la soluzione sia del tipo
$$
e^{\lambda t}\cdot c\quad \lambda \in \mathbb{R}, c\in \mathbb{R}^n
$$

Sostituendo l'ipotesi di soluzione nel sistema e differenziando si ottiene
$$
\lambda \cancel{e^{\lambda t}} c = A\cancel{e^{\lambda t}} c
$$
$$
\lambda c = A c \Leftrightarrow (\lambda I -A)c = 0
$$
Questa equazione è vera solo se $\lambda$ è un autovalore (con $c\neq
\underline{0}$).
Esistono solo $n$ soluzioni possibili (nell'ipotesi che la matrice $A$ sia di
semplice struttura, non è un'ipotesi necessaria).

Qualunque soluzione $x_h$ combinazione lineare delle soluzioni precedentemente
ricavata, è ancora soluzione del sistema.
$$
x_h(t) = \sum_{i=1}^n e^{\lambda_i t} c_i
$$
Analogamente si potrebbe procedere con la soluzione particolare.

\subsection{Matrice esponenziale}
Si indica simbolicamente con
\begin{equation}
e^{At} \stackrel{\text{def}}{=} \sum_{k=0}^{+\infty} \frac{A^k t^k}{k!}
\end{equation}
È una serie nel tempo e non una matrice di esponenziali.

Si cercano le ipotesi di convergenza della serie, la sua norma è maggiorata da
una serie assolutamente convergente (per ciascun valore di $t$)
$$
\left|\left|e^{At}\right|\right| = \left|\left|\sum_{k} \frac{A^k
t^k}{k!}\right|\right| \leq \sum_k \frac{\left|\left|A^k\right|\right| t^k}{k!}
\leq \sum_k \frac{a^kt^k}{k!} \stackrel{\text{def}}{=}e^{at}
$$
La norma della somma è minore o uguale alla somma delle norme e i tempi sono
assunti positivi.
Si ricava il coefficiente $a^k$ dato che $||A^k|| \leq ||A||^k = a^k $, quella
ottenuta è la definizione di esponenziale scalare.

Si vuole scrivere una soluzione analitica della matrice esponenziale, si
supponga che sia diagonalizzabile per similitudine, ossia se ne possa costruire
la matrice modale, ciò è sicuramente vero se la matrice è di semplice struttura.
In generale è sufficiente che per ogni autovalore la molteplicità
algebrica coincida con quella geometrica, ossia siano presenti $n$ autovettori
distinti (esista una base).
Si può costruire la matrice $\Lambda$ degli autovalori
$$
\Lambda = U^{-1}AU \Leftrightarrow A=U\Lambda U^{-1}
$$
Si eleva la matrice $A$ a potenza
$$
A^k = (U\Lambda U^{-1})^k = U\Lambda U^{-1} \cdot U\Lambda U^{-1} \cdot
\ldots\cdot U\Lambda U^{-1}\ k\ \text{volte}
$$
Si nota che sono sempre presenti $k$ prodotti $U^{-1}U$ pari alla matrice
identità, dunque
$$
A^k = U\Lambda^kU^{-1}
$$
Ma lambda è una matrice diagonale, moltiplicata per sé stessa fornisce ancora
una matrice diagonale
\begin{equation}
A^k = U \begin{bmatrix}
        \lambda_1^k & 0 & 0      \\
        \vdots & \ddots & \vdots \\
        0 & 0 & \lambda_n^k
        \end{bmatrix} U^{-1} =
\begin{bmatrix}u_1 & \ldots & u_n\end{bmatrix} \begin{bmatrix}
        \lambda_1^k & 0 & 0      \\
        \vdots & \ddots & \vdots \\
        0 & 0 & \lambda_n^k
        \end{bmatrix}
        \begin{bmatrix}
        v_1^T\\ \vdots \\ v_n^T
        \end{bmatrix}
\label{eq.:matrice_esponenziale}
\end{equation}
Sviluppando i prodotti
\begin{equation}
A^k = \sum_{i=1}^n \lambda_i^k u_i v_i^T
\label{eq.:forma_spettrale}
\end{equation}
Si ottiene la somma di $n$ matrici $n\times n$ denominate residui
polari $R_i$ moltiplicate per $\lambda^k$.
Una proprietà dei residui polari, nel caso di matrice diagonalizzabile, è che
hanno rango pari ad uno.

La \ref{eq.:forma_spettrale} prende il nome di \textit{forma spettrale} (o
forma \textit{diadica}) della matrice potenza.

\subsubsection{Calcolo della matrice esponenziale}
Si supponga che $A$ sia diagonalizzabile, si vuole calcolare la matrice
esponenziale
$$
e^{At} \stackrel{\text{def}}{=} \sum_{k=0}^{+\infty} \frac{A^kt^k}{k!}
$$

52:54
