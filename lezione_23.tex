
\subsubsection{Polinomi con coefficienti discordi}
Si consideri il seguente polinomio
$$
p(s) = s^5 + 8s^4 + 17s^3 + 8s^2 -14s - 10
$$
La condizione necessaria non è soddisfatta, il polinomio non può rappresentare
un sistema asintoticamente stabile, esisteranno delle radici la cui parte reale
sarà maggiore o uguale a zero.
Si costruisce comunque la tabella di Routh per esercizio e per contare il
numero di radici positive.
\begin{table}[h]
$$
\begin{array}{c:c|ccc}
& 5 & 1 & 17 & -4 \\
p & 4 & 8 & 8 & -10 \\ \hline
p & 3 & 16 &-23/2 \\
p & 2 & 55/4 & -20 \\
p & 1 & 259/22 \\
c & 0 & -20
\end{array}
$$
\end{table}
Sono presenti quattro permanenze di segno consecutive mentre il passaggio dalla
riga 1 alla riga 0 contiene una variazione di segno. Il polinomio avrà una sola
radice a parte reale positiva, la presenza del termine noto -10 esclude la
presenza di uno zero a parte reale nulla.
Risolvendo il polinomio con un calcolatore si ottiene:
$$
\lambda_i = \{ -5,-2,-1\pm j,1 \}
$$

\subsubsection{Ulteriore esempio}
Anche in questo caso non è soddisfatta la condizione necessaria
$$
p(s) = s^4 - 6s^3 - 11s^2 + 74 s + 78
$$
Si applica la tabella di Routh
$$
\begin{array}{c:c|ccc}
&4 & 1 & -11 & 78\\
c&3 & -6 & 74 \\ \hline
c&2 & 4/3 & 78 \\
p&1 & 425 \\
p&0 & 78
\end{array}
$$
Il polinomio ha due radici a parte reale negativa e due a parte reale maggiore
o uguale a zero
$$
\lambda_i = \{-3,-1,5\pm j\}
$$

\newpage
\subsubsection{Tabella contenente uno zero}
Si analizza il seguente polinomio
$$
p(s) = s^4 + s^3 + s^2 + s + 2
$$
È soddisfatta la condizione necessaria ma nella costruzione della tabella
compare uno zero, dunque non è ben definita e il polinomio non ammette tutte le
soluzioni a parte reale negativa.
$$
\begin{array}{c:c|cccc}
 & 4 & 1 & 1 & 2 \\
 & 3 & 1 & 1 \\ \hline
 & 2 & 0  \\
 & 1 \\
 & 0
\end{array}
$$

Se si volesse comunque contare il numero di zeri a parte reale positiva e
negativa esistono alcune tecniche equivalenti per continuare il calcolo.

\textbf{Tecnica delle perturbazioni elementari}

Si può sommare una $\varepsilon$ infinitesima positiva ai coefficienti, ossia
perturbarli localmente per eliminare lo zero, si continua lo sviluppo della
tabella:
$$
\begin{array}{cc:c|cccc}
\varepsilon<0 & \varepsilon>0  & 4 & 1 & 1 & 2 \\
p & p & 3 & 1 & 1 \\ \hline
c & p & 2 & \varepsilon & 2  \\
c & c & 1 & -2/\varepsilon\\
p & c & 0 & 2
\end{array}
$$
Il primo elemento sotto lo zero sarà
$$
c_{1,1} = -\frac{1}{\varepsilon}\begin{vmatrix}
1 & 1 \\
\varepsilon & 2
\end{vmatrix} = -\frac{1}{\varepsilon}(2-\varepsilon) \simeq
-\frac{2}{\varepsilon}
$$
Con $\varepsilon$ infinitesimo positivo si hanno due permanenze e due cambi di
segno, se $\varepsilon$ fosse negativo invece si vede che il risultato non
cambia.
Questa tecnica è valida se si azzera un elemento in prima colonna ma non
più di uno zero per riga.

\subsubsection{Zeri multipli sulla stessa riga}
Si consideri una riga $i$-esima contenente più di uno zero tra i primi $p$
elementi, si moltiplicano i coefficienti della riga per $(-1)^p$,
successivamente si trasla la riga verso sinistra di $p$ posizioni eliminando
gli zeri, si somma questa nuova riga più piccola ottenuta e la si somma alla
riga $i$-esima originale; il risultato della somma viene sostituito al posto
della riga $i$-esima.

Si applica la tecnica al polinomio precedente, si vede che $p=1$, la riga 2
18:40
