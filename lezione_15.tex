
\section{Equazione differenziale generale}
Sia un sistema LTI espresso mediante la sua equazione differenziale generale
$$
y^{(n)} + \alpha_{n-1}y^{(n-1)} + \alpha_{n-2}y^{(n-2)} + \ldots +
\alpha_0 y = \beta_mu^{(m)}+ \beta_{m-1}u^{(m-1)} +
\ldots + \beta_0 u
$$
Si vuole conoscere il legame tra questa equazione e la funzione di
trasferimento, è necessario conoscere anche le condizioni iniziali, si può
supporre che siano tutte nulle:
$$
y(0) = \dot{y}(0) = \ldots = y^{(n-1)}(0) = 0 \qquad
[y(t)=y_f(t)]
$$

Si trasforma l'intera equazione con Laplace
$$\begin{aligned}
s^nY(s)+\alpha_{n-1}s^{n-1}Y(s) + \ldots + \alpha_0 Y(s) &= \beta_m s^m U(s) +
\ldots + \beta_0U(s)\\
\left(s^n+\alpha_{n-1}s^{n-1} + \ldots + \alpha_0 \right)Y(s) &= \left(\beta_m
s^m  +
\ldots + \beta_0\right)U(s)
\end{aligned}$$
Nel caso di sistemi \textit{SISO} la funzione di trasferimento si ricava con il
rapporto tra le trasformate dell'uscita e dell'ingresso, utilizzando la
precedente si ottiene
$$
W(s) = \frac{Y_f(s)}{U(s)} = \frac{\beta_m
s^m  +
\ldots + \beta_0}{s^n+\alpha_{n-1}s^{n-1} + \ldots +
\alpha_0}
$$
Nei sistemi LTI la funzione di trasferimento si esprime sempre in una forma
razionale fratta. L'ordine $n$ del polinomio al denominatore sarà sempre
maggiore o uguale a quello del numeratore, viceversa si avrebbe un sistema non
causale, non di interesse nell'ambito del corso.
antitrasformando la funzione di trasferimento si può ricavare l'equazione
differenziale generale.
08:11
