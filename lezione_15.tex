
\subsection{Equazione differenziale generale}
Sia un sistema LTI espresso mediante la sua equazione differenziale generale
$$
y^{(n)} + \alpha_{n-1}y^{(n-1)} + \alpha_{n-2}y^{(n-2)} + \ldots +
\alpha_0 y = \beta_mu^{(m)}+ \beta_{m-1}u^{(m-1)} +
\ldots + \beta_0 u
$$
Si vuole conoscere il legame tra questa equazione e la funzione di
trasferimento, è necessario conoscere anche le condizioni iniziali, si può
supporre che siano tutte nulle:
$$
y(0) = \dot{y}(0) = \ldots = y^{(n-1)}(0) = 0 \qquad
[y(t)=y_f(t)]
$$

Si trasforma l'intera equazione con Laplace
$$\begin{aligned}
s^nY(s)+\alpha_{n-1}s^{n-1}Y(s) + \ldots + \alpha_0 Y(s) &= \beta_m s^m U(s) +
\ldots + \beta_0U(s)\\
\left(s^n+\alpha_{n-1}s^{n-1} + \ldots + \alpha_0 \right)Y(s) &= \left(\beta_m
s^m  +
\ldots + \beta_0\right)U(s)
\end{aligned}$$
Nel caso di sistemi \textit{SISO} la funzione di trasferimento si ricava con il
rapporto tra le trasformate dell'uscita e dell'ingresso, utilizzando la
precedente si ottiene
$$
W(s) = \frac{Y_f(s)}{U(s)} = \frac{\beta_m
s^m  +
\ldots + \beta_0}{s^n+\alpha_{n-1}s^{n-1} + \ldots +
\alpha_0}
$$
Nei sistemi LTI la funzione di trasferimento si esprime sempre in una forma
razionale fratta. L'ordine $n$ del polinomio al denominatore sarà sempre
maggiore o uguale a quello del numeratore, viceversa si avrebbe un sistema non
causale, non di interesse nell'ambito del corso.
antitrasformando la funzione di trasferimento si può ricavare l'equazione
differenziale generale e viceversa.

\subsection{Struttura della funzione di trasferimento}
Si ricorda che l'unico termine dipendente da $s$ nella funzione di
trasferimento è la trasformata della funzione di transizione $\Phi(s)$
$$
W(s) = C(sI-A)^{-1} B +D = C\Phi(s) B+D
$$
Si analizza la struttura di $\Phi(s)$, l'inversione di una matrice si esegue
con il rapporto della trasposta della matrice \textit{aggiunta} e il suo
determinante
$$
\Phi(s) = (sI-A)^{-1} = \frac{\left(\left(sI-A\right)^a\right)^T}{|sI-A|}
$$
Il denominatore è il polinomio caratteristico.
\newpage
In misura esemplificativa si considera una matrice $2\times2$
$$\begin{aligned}
A&=\begin{pmatrix}
   a_{11} & a_{12} \\
   a_{21} & a_{22}
  \end{pmatrix}\\
  \Phi(s) &=
\begin{pmatrix}
 s-a_{11} & -a_{12} \\
 -a_{21} & s-a_{22}
\end{pmatrix}^{-1} =
\frac{\left(\begin{pmatrix}
 s-a_{11} & -a_{12} \\
 -a_{21} & s-a_{22}
\end{pmatrix}^a\right)^T}{(s-a_{11})(s-a_{22})-a_{12}a_{21}}
= \frac{\begin{pmatrix}
           s-a_{22} & a_{12} \\
           a_{21} & s-a_{11}
         \end{pmatrix}
}{p(s)}
\end{aligned}$$
L'ordine del polinomio è due mentre l'ordine dei polinomi contenuti nella
matrice al numeratore sono al più di ordine $n-1$, questo è sempre vero dato
che i complementi algebrici si costruiscono rimuovendo una riga ed
una colonna quindi riducendo il grado della matrice di partenza, di conseguenza
il grado del suo determinante.

Tornando al caso generale
$$
\Phi(s) = \frac{E(s)}{p(n)}
$$
I polinomi contenuti nella matrice $E(s)$ possono essere fattorizzati
e potrebbe capitare che tutti abbiano uno o più fattori in comune, possono
essere messi in evidenza. Qualcuno di questi fattori potrebbe essere in comune
anche ai fattori del denominatore, si potrebbero dunque semplificare, in tal
caso il polinomio al denominatore indicato con $m(s)$ prende il nome di
\textit{polinomio minimo} con grado minore o uguale ad $n$.
Nel caso in cui non ci siano state semplificazioni si parla di sistema in
\textit{forma minima}.

La struttura generale del polinomio caratteristico:
$$
p(s) = |sI-A| = \prod_{i=1}^{n'}(s-\lambda_i)^{m_{ai}}
$$
La struttura corrispondente del polinomio minimo si può dimostrare essere:
$$
m(s) = \prod_{i=1}^{n'} (s-\lambda_i)^{m_{gi}}
$$
Le radici sono ancora tutte presenti ma stavolta non presenti con esponente la
molteplicità algebrica bensì quella geometrica.

Si ricostruisce la matrice $W(s)$ e si indicano le dimensioni delle matrici che
la compongono:
$$\begin{array}{c c c c c c c}
W(s) &= &C&\Phi(s)&B &+ &D\\
p\times m & & p\times n& n \times n & n \times m & & p \times m
\end{array}$$
La matrice $\Phi(s)$ è composta da funzioni razionali fratte, tutte
strettamente proprie, viene moltiplicata a destra e a sinistra da matrici
costanti, sarà ancora formata da funzioni razionali fratte in $s$ strettamente
proprie. Quando si somma con la matrice $D$, se questa non è nulla, si
otterranno funzioni razionali fratte proprie, in cui il grado di numeratore e
denominatore coincidono.

\subsection{Trasformazione del sistema in forma di stato}
Si ricorda che
$$
\Phi(s) = (sI-A)^{-1} = \Lap\left[e^{At}\right]
\stackrel{\Lap^{-1}}{\longrightarrow}  e^{At} = \Lap^{-1}[\Phi(s)] =
\Lap^{-1}\left[(sI-A)^{-1}\right]
$$

\textbf{Esempio numerico:}
$$
A = \begin{pmatrix}
     0 & 1 \\ -2 & -3
    \end{pmatrix}
\longrightarrow (sI-A) =
\begin{pmatrix}
 s & -1 \\ 2 & s+3
\end{pmatrix}
$$
Si calcola l'inversa
$$
\Phi(s) = (sI-A)^{-1} = \frac{
\begin{bmatrix}
  s+3 & 1 \\ -2 & s
  \end{bmatrix}
}{\begin{aligned}
&s(s+3)+2\\
&(s+1)(s+2)
  \end{aligned}
} =
\frac{R_1}{s+1} + \frac{R_2}{s+2}
$$
Scomponendo in fratti semplici, in questo caso i residui saranno matrici
$2\times 2$.
$$
R_1 = \lim_{s\to -1} (s+1) \Phi(s) = \lim_{s\to -1} \frac{
\begin{bmatrix}
 s+3 & 1 \\ -2 & s
\end{bmatrix}
}{s+2} =
\frac{
\begin{bmatrix}
 2 & 1 \\ -2 & -1
\end{bmatrix}
}{1} = \begin{bmatrix}
 2 & 1 \\ -2 & -1
\end{bmatrix}
$$

Analogamente il secondo residuo
$$
R_2 = \lim_{s\to -2} (s+2) \Phi(s) = \lim_{s\to -2} \frac{
\begin{bmatrix}
 s+3 & 1 \\ -2 & s
\end{bmatrix}
}{s+1} =
\frac{
\begin{bmatrix}
 1 & 1 \\ -2 & -2
\end{bmatrix}
}{-1} = \begin{bmatrix}
 -1 & -1 \\ 2 & 2
\end{bmatrix}
$$
Entrambe le matrici sono di rango unitario pari alla molteplicità
algebrica della corrispondente radice $\rho(R_i) = m_{ai}$.

Sostituendo si ottiene
$$
\Phi(s) = \frac{1}{s+1}\begin{bmatrix}
                        2 & 1 \\ -2 & -1
                       \end{bmatrix}
+ \frac{1}{s+2}
\begin{bmatrix}
 -1 & -1 \\ 2 & 2
\end{bmatrix}
$$
41:00
