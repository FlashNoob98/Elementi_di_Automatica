
\subsection{Equazione differenziale generale}
Sia un sistema LTI espresso mediante la sua equazione differenziale generale
$$
y^{(n)} + \alpha_{n-1}y^{(n-1)} + \alpha_{n-2}y^{(n-2)} + \ldots +
\alpha_0 y = \beta_mu^{(m)}+ \beta_{m-1}u^{(m-1)} +
\ldots + \beta_0 u
$$
Si vuole conoscere il legame tra questa equazione e la funzione di
trasferimento, è necessario conoscere anche le condizioni iniziali, si può
supporre che siano tutte nulle:
$$
y(0) = \dot{y}(0) = \ldots = y^{(n-1)}(0) = 0 \qquad
[y(t)=y_f(t)]
$$

Si trasforma l'intera equazione con Laplace
$$\begin{aligned}
s^nY(s)+\alpha_{n-1}s^{n-1}Y(s) + \ldots + \alpha_0 Y(s) &= \beta_m s^m U(s) +
\ldots + \beta_0U(s)\\
\left(s^n+\alpha_{n-1}s^{n-1} + \ldots + \alpha_0 \right)Y(s) &= \left(\beta_m
s^m  +
\ldots + \beta_0\right)U(s)
\end{aligned}$$
Nel caso di sistemi \textit{SISO} la funzione di trasferimento si ricava con il
rapporto tra le trasformate dell'uscita e dell'ingresso, utilizzando la
precedente si ottiene
$$
W(s) = \frac{Y_f(s)}{U(s)} = \frac{\beta_m
s^m  +
\ldots + \beta_0}{s^n+\alpha_{n-1}s^{n-1} + \ldots +
\alpha_0}
$$
Nei sistemi LTI la funzione di trasferimento si esprime sempre in una forma
razionale fratta. L'ordine $n$ del polinomio al denominatore sarà sempre
maggiore o uguale a quello del numeratore, viceversa si avrebbe un sistema non
causale, non di interesse nell'ambito del corso.
antitrasformando la funzione di trasferimento si può ricavare l'equazione
differenziale generale e viceversa.

\subsection{Struttura della funzione di trasferimento}
Si ricorda che l'unico termine dipendente da $s$ nella funzione di
trasferimento è la trasformata della funzione di transizione $\Phi(s)$
$$
W(s) = C(sI-A)^{-1} B +D = C\Phi(s) B+D
$$
Si analizza la struttura di $\Phi(s)$, l'inversione di una matrice si esegue
con il rapporto della trasposta della matrice \textit{aggiunta} e il suo
determinante
$$
\Phi(s) = (sI-A)^{-1} = \frac{\left(\left(sI-A\right)^a\right)^T}{|sI-A|}
$$
Il denominatore è il polinomio caratteristico.
\newpage
In misura esemplificativa si considera una matrice $2\times2$
$$\begin{aligned}
A&=\begin{pmatrix}
   a_{11} & a_{12} \\
   a_{21} & a_{22}
  \end{pmatrix}\\
  \Phi(s) &=
\begin{pmatrix}
 s-a_{11} & -a_{12} \\
 -a_{21} & s-a_{22}
\end{pmatrix}^{-1} =
\frac{\left(\begin{pmatrix}
 s-a_{11} & -a_{12} \\
 -a_{21} & s-a_{22}
\end{pmatrix}^a\right)^T}{(s-a_{11})(s-a_{22})-a_{12}a_{21}}
= \frac{\begin{pmatrix}
           s-a_{22} & a_{12} \\
           a_{21} & s-a_{11}
         \end{pmatrix}
}{p(s)}
\end{aligned}$$
L'ordine del polinomio è due mentre l'ordine dei polinomi contenuti nella
matrice al numeratore sono al più di ordine $n-1$, questo è sempre vero dato
che i complementi algebrici si costruiscono rimuovendo una riga ed
una colonna quindi riducendo il grado della matrice di partenza, di conseguenza
il grado del suo determinante.

Tornando al caso generale
$$
\Phi(s) = \frac{E(s)}{p(n)}
$$
I polinomi contenuti nella matrice $E(s)$ possono essere fattorizzati
17:12
