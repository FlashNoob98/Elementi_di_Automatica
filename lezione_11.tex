\section{Modi naturali per i sistemi dinamici}
I modi naturali sono tutte le funzioni che possono comparire nella risposta in
evoluzione libera, ovvero tutte le funzioni del tempo che possono essere
identificate all'interno della matrice esponenziale
$$
x_l(t) = e^{At}x_0\ x_0 \in \mathbb{X}^n
$$

Si consideri un sistema con $n=3$ con i seguenti autovalori e autovettori della
matrice della dinamica $A$:
$$\begin{array}{ccc}
\lambda & \alpha+j\omega & \alpha-j\omega\\
\downarrow &\downarrow &\downarrow \\
u & u_a +j u_b & u_a -j u_b
\end{array}$$

Si costruisce la matrice modale reale $U_r$ e la sua inversa $U_r^{-1}$
$$\begin{aligned}
U_r &= [u \ u_a \ u_b] \\
U_r^{-1} &= \begin{bmatrix}
             v^T\\v_a^T\\v_b^T
            \end{bmatrix}
\end{aligned}$$

Si introduce il cambio di variabili con una nuova variabile di stato $z$
$$
z = U_r^{-1} x\rightarrow \Lambda_r = U_r^{-1}AU_r = \left(
\begin{array}{c:cc}
 \lambda & 0 & 0 \\ \hdashline
 0 & \alpha & \omega \\
 0 & -\omega & \alpha
\end{array}
\right)
$$

Calcolando la matrice esponenziale si ottiene
$$
e^{At} = e^{\left(U_r\Lambda_r U_r^{-1} \right)t} = U_r e^{\Lambda_r t}U_r^{-1}
= \begin{pmatrix}
   u & u_a & u_b
  \end{pmatrix}\left(
\begin{array}{c:c}
 e^{\lambda t}& 0 \quad 0 \\ \hdashline
\begin{aligned}
0 \\ 0
\end{aligned} &\begin{aligned} \\ e^{\begin{pmatrix}
\alpha& \omega\\
-\omega & \alpha
\end{pmatrix}t}\end{aligned}
\end{array}\right)
\begin{pmatrix}
v^T \\ v_a^T \\ v_b^T
\end{pmatrix}
$$
Sviluppando l'esponenziale interno
$$
e^{\begin{pmatrix}
\alpha& \omega\\
-\omega & \alpha
\end{pmatrix}t} =
e^{\alpha t} \begin{pmatrix}
              \cos \omega t & \sin \omega t \\
              -\sin \omega t & \cos \omega t
             \end{pmatrix}
$$
La matrice esponenziale completa dunque:
$$
e^{At} = \begin{pmatrix}
   u & u_a & u_b
  \end{pmatrix}
\begin{pmatrix}
e^{\lambda t} & 0 & 0 \\
0 & e^{\alpha t}\cos\omega t & e^{\alpha t}\sin\omega t \\
0 & -e^{\alpha t}\sin\omega t & e^{\alpha t}\cos\omega t
\end{pmatrix}
\begin{pmatrix}
v^T \\ v_a^T \\ v_b^T
\end{pmatrix}
$$
Ricordando la definizione di modo naturale si vede che le funzioni del tempo
sono tutte contenute nella matrice esponenziale degli autovalori
$e^{\Lambda_r t}$

Si sviluppano i prodotti
$$
e^{At} = e^{\lambda t}uv^T + e^{\alpha t}u_a \left(\cos(\omega t) v_a^T +
\sin(\omega t)v_b^T \right)+
e^{\alpha t}u_b \left(-\sin(\omega t) v_a^T +
\cos(\omega t)v_b^T \right)
$$

Si considera ora l'evoluzione libera
$$\begin{aligned}
x_l(t) &= e^{At}x_0 = e^{\lambda t} uv^Tx_0 + e^{\alpha t}\left[
u_a \left(\cos(\omega t) v_a^Tx_0 +
\sin(\omega t)v_b^Tx_0 \right) +\right. \\
 &\left.+ e^{\alpha t}u_b \left(-\sin(\omega t) v_a^Tx_0 +
\cos(\omega t)v_b^Tx_0 \right)
\right]
\end{aligned}$$
si osserva che i vettori $v^T$ sono righe mentre $x_0$ sono vettori colonna, di
conseguenza il loro prodotto è uno scalare $c$, si fanno le seguenti posizioni
$$
\begin{bmatrix}
v^T \\ v_a^T \\ v_b^T
\end{bmatrix}
x_0 =
\begin{pmatrix}
 c \\ c_a \\  c_b
\end{pmatrix}
$$
Si riscrive l'equazione dell'evoluzione libera
$$\begin{aligned}
x_l(t) &= ce^{\lambda t}u +e^{\alpha t} u_a\left(\cos(\omega t)c_a +
\sin(\omega t) c_b\right) + \\
& + e^{\alpha t} u_b\left(-\sin(\omega t)c_a +
\cos(\omega t) c_b\right)
\end{aligned}$$
Se si analizza lo stato nell'istante di tempo nullo
$$
x_l(0) = cu + c_au_a +c_bu_b =x_0 \Rightarrow z_l(0) = \begin{pmatrix}
 c \\ c_a \\  c_b
\end{pmatrix} = z_0 = U_R^{-1} x_0
$$
Si osserva che le costanti $c$ sono proprio le componenti dello stato iniziale
rispetto alla nuova base $u$.

Si considerino le seguenti posizioni per definire l'angolo $\varphi$
$$
m=\sqrt{c_a^2 + c_b ^2} \qquad \sin\varphi = \frac{c_a}{m} \qquad \cos\varphi =
\frac{c_b}{m}
$$
si riscrive la $x_l(t)$ in forma ancora più compatta, sfruttando le formule di
prostaferesi
$$
x_l(t) = ce^{\lambda t} u + me^{\alpha t}\left(\sin(\omega t +\varphi)u_a +
\cos(\omega t + \varphi)u_b\right)
$$
Le funzioni dipendenti dal tempo, generate da matrici diagonalizzabili, possono
essere al più esponenziali scalari per ogni autovalore reale, chiamati moti
\textbf{aperiodici}; oppure funzioni sinusoidali moltiplicate per esponenziali
per ogni autovalore complesso e coniugato, verranno detti modi
\textbf{pseudo-periodici}.

Si vuole capire come sarà l'andamento qualitativo dell'evoluzione libera dello
stato al variare degli autovalori.

\newpage
\subsection{Moti aperiodici}
Sono generati da autovalori reali
$$
\lambda \text{ reali} \rightarrow e^{\lambda t} u
$$
Si possono avere tre casi
\begin{figure}[h]
\centering
 \begin{subfigure}[b]{0.32\textwidth}
  \centering
  \begin{tikzpicture}
   \begin{axis}[
     axis lines=left,
     width=\textwidth,
     xtick={0,3.5},
     xticklabels={0,t},
     ytick={1,1.5},
     ymax=1.5,
     yticklabels={1,$e^{\lambda t}$},
     ]
    \addplot[color=black,domain=0:3.5]{exp(-x)};
   \end{axis}
  \end{tikzpicture}
  \caption{$\lambda<0$}
 \end{subfigure}
\hfill
 \begin{subfigure}[b]{0.32\textwidth}
  \centering
  \begin{tikzpicture}
   \begin{axis}[
     axis lines=left,
     width=\textwidth,
     xtick={0,3.5},
     xticklabels={0,t},
     ytick={1,1.5},
     ymax=1.5,
     ymin = 0,
     yticklabels={1,$e^{\lambda t}$},
     ]
    \addplot[color=black,domain=0:3.5]{1};
   \end{axis}
  \end{tikzpicture}
  \caption{$\lambda=0$}
 \end{subfigure}
\hfill
 \begin{subfigure}[b]{0.32\textwidth}
  \centering
  \begin{tikzpicture}
   \begin{axis}[
     axis lines=left,
     width=\textwidth,
     xtick={0,3.5},
     xticklabels={0,t},
     ytick={1,34},
     ymax=34,
     ymin = 0.5,
     yticklabels={1,$e^{\lambda t}$},
     ]
    \addplot[color=black,domain=0:3.5]{exp(x)};
   \end{axis}
  \end{tikzpicture}
  \caption{$\lambda>0$}
 \end{subfigure}
\end{figure}

