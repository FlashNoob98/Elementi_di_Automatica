%Lezione 01
\chapter{Introduzione}
L'automazione è una disciplina estremamente ampia, si intende in questo corso con
\textit{automazione} la progettazione, la realizzazione e la gestione di sistemi in grado di
eseguire dei compiti in maniera autonoma, senza l'intervento dell'uomo.

Nel corso verranno effettuate le analisi di sistemi dinamici, con un focus finale sulle analisi di
sistemi di controllo che verranno approfondite al corso di ``controlli''.

L'uomo ha sempre cercato di automatizzare i processi o i compiti che doveva eseguire, per ridurre
la fatica e l'usura per le attività manuali.

Un esempio è il \textbf{regolatore di Watt}, una macchina in grado di regolare il grado di
ammissione di una valvola di alimentazione per una macchina a vapore, al fine di mantenere costante
la velocità di rotazione della macchina.

Inizialmente l'operazione era compiuta da un operatore che regolava la temperatura della caldaia
fornendo più o meno combustibile.

Quest'oggetto racchiude l'essenza dei controlli automatici:
\begin{itemize}
 \item Elemento di trasduzione e misura, fondamentale per ottenere informazioni sulla grandezza da
controllare, in questo caso la velocità di rotazione delle macchine da controllare. L'uscita dello
strumento di misura può essere di diversa natura rispetto alla grandezza misurata ma comunque
proporzionale ad esso.
 \item Elemento di controllo (controllore), in questo caso il sistema di leve e pesi che varia la
posizione del cursore in funzione dell'input e delle sue caratteristiche come i pesi e le lunghezze
delle leve.
\item Attuatore, ossia uno strumento in grado di attuare la decisione del controllore sul sistema,
nel caso precedente la valvola.
\end{itemize}

Nel contesto più generale dei sistemi dinamici si riuscirà a modellare ed analizzare i sistemi
dinamici in generale e comprendere le proprietà fondamentali e strutturali dei sistemi studiati.

\section{Sistemi dinamici}
Un sistema è qualunque oggetto o processo materiale o immateriale, ben delimitato nel suo
funzionamento.
Potrebbe essere un sistema meccanico o termico, o ad esempio un sistema immateriale come
l'andamento del PIL in Italia.

Gli oggetti di interesse in particolare sono quelli \textit{dinamici}
che hanno ossia la possibilità di variare nel tempo alcune grandezze che li caratterizzano.

L'unica variabile indipendente considerata nell'intero corso sarà il tempo, anche i sistemi
astratti saranno comunque sistemi ``esistenti''.

Dato un certo sistema si dovrà astrarre dal sistema un modello che lo rappresenta,
operazione denominata modellistica, il modello prende il nome di \textbf{oggetto astratto
orientato}.

Con ``identificazione dei sistemi dinamici'' si intende la scienza che permette la costruzione di
modelli matematici anche quando i sistemi in esame e le loro proprietà fisiche non sono note.

Costruito il modello è poi possibile analizzare le caratteristiche dell'oggetto, attraverso la
quale si possono conoscere le caratteristiche comportamentali del modello.

Le previsioni meteorologiche sono un classico esempio di analisi in un sistema dinamico, va
modellato il pianeta in funzione del fenomeno da studiare, vanno quindi determinate le variabili
inerenti il fenomeno. Esiste un modello matematico del pianeta che permette di prevedere le
grandezze future del sistema dato lo stato attuale dello stesso.

Se si esegue ad esempio l'analisi del modello termico di una stanza si ottiene un valore di
temperatura diverso da quello desiderato e corrispondente al comportamento naturale del sistema.
Se si desidera un valore di temperatura differente, va costruito un sistema di condizionamento e
controllo che con un attuatore modifichi l'evoluzione del sistema.

\section{Oggetto astratto orientato}
È un'astrazione di un oggetto reale, dal quale si ricavano un insieme di equazioni e variabili che
ne descrivano le grandezze di interesse.
È un modello \textit{parziale} della realtà, sia perché analizza solo uno o una parte di fenomeno,
sia perché, anche nella grandezza di interesse, non sarà mai identico alla realtà a causa delle
varie approssimazioni.

L'accuratezza di un modello solitamente aumenta all'aumentare della complessità del modello, va
determinato il grado di complessità in funzione dell'accuratezza richiesta al risultato.

Con il termine \textit{orientato} si intende sottolineare che il modello matematico deve essere un
modello \textbf{causale}, ossia che rispetti il principio di causa-effetto, l'effetto
\textit{segue} nel tempo la causa, altrimenti sarebbe un sistema che prevede il futuro. In alcune
discipline si studiano anche sistemi non causali, come ad esempio nella trasmissione digitale
terrestre possono essere utilizzati dei filtri anti-causali, trasmettendo ad esempio il segnale con
una latenza di tre secondi, è possibile sfruttare questi secondi di ``buffer'' per ricostruire in
maniera più accurata il segnale.

Il concetto di ``orientato'' fa nascere un'idea discriminante nell'ambito delle variabili presenti
nel modello, dividendole in due grandi \textit{famiglie}, ossia
\textbf{ingresso} o \textbf{uscita}.

