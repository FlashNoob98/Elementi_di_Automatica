%Lezione 01
\chapter{Introduzione}
L'automazione è una disciplina estremamente ampia, si intende in questo corso con
\textit{automazione} la progettazione, la realizzazione e la gestione di sistemi in grado di
eseguire dei compiti in maniera autonoma, senza l'intervento dell'uomo.

Nel corso verranno effettuate le analisi di sistemi dinamici, con un focus finale sulle analisi di
sistemi di controllo che verranno approfondite al corso di ``controlli''.

L'uomo ha sempre cercato di automatizzare i processi o i compiti che doveva eseguire, per ridurre
la fatica e l'usura per le attività manuali.

Un esempio è il \textbf{regolatore di Watt}, una macchina in grado di regolare il grado di
ammissione di una valvola di alimentazione per una macchina a vapore, al fine di mantenere costante
la velocità di rotazione della macchina.

Inizialmente l'operazione era compiuta da un operatore che regolava la temperatura della caldaia
fornendo più o meno combustibile.

Quest'oggetto racchiude l'essenza dei controlli automatici:
\begin{itemize}
 \item Elemento di trasduzione e misura, fondamentale per ottenere informazioni sulla grandezza da
controllare, in questo caso la velocità di rotazione delle macchine da controllare. L'uscita dello
strumento di misura può essere di diversa natura rispetto alla grandezza misurata ma comunque
proporzionale ad esso.
 \item Elemento di controllo (controllore), in questo caso il sistema di leve e pesi che varia la
posizione del cursore in funzione dell'input e delle sue caratteristiche come i pesi e le lunghezze
delle leve.
\item Attuatore, ossia uno strumento in grado di attuare la decisione del controllore sul sistema,
nel caso precedente la valvola.
\end{itemize}

Nel contesto più generale dei sistemi dinamici si riuscirà a modellare ed analizzare i sistemi
dinamici in generale e comprendere le proprietà fondamentali e strutturali dei sistemi studiati.

\section{Sistemi dinamici}
Un sistema è qualunque oggetto o processo materiale o immateriale, ben delimitato nel suo
funzionamento.
Potrebbe essere un sistema meccanico o termico, o ad esempio un sistema immateriale come
l'andamento del PIL in Italia.

Gli oggetti di interesse in particolare sono quelli \textit{dinamici}
che hanno ossia la possibilità di variare nel tempo alcune grandezze che li caratterizzano.

L'unica variabile indipendente considerata nell'intero corso sarà il tempo, anche i sistemi
astratti saranno comunque sistemi ``esistenti''.

33:04
