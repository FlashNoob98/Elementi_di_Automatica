
\subsection{Esempi numerici trasformate fratte}
\subsubsection{Poli distinti}
Sia la seguente funzione razionale fratta con due poli distinti: $p_1 = -2$,
$p_2=-5$, $n=2,\ m=1,\ n-m=1$
$$
F(s) = \frac{s-10}{(s+2)(s+5)} = \frac{R_1}{s+2} + \frac{R_2}{s+5}
$$

Si calcolano i residui polari:
$$\begin{aligned}
R_1 &= \lim_{s\to -2} (s+2)F(s) = \lim_{s\to -2} \frac{s-10}{s+5} =
\frac{-12}{+3} = -4\\
R_2 &= \lim_{s\to -5} (s+5)F(s) = \lim_{s\to -5}
\frac{s-10}{s+2} = \frac{-15}{-3} = 5\\
\sum_{i=1}^2R_i &= \frac{b_m}{a_n} = \frac{1}{1} \Rightarrow R_2 = 1-R_1
\end{aligned}
$$
L'antitrasformata:
$$
f(t) = \Lap^{-1}[F(s)] = \Lap^{-1}\left[-\frac{4}{s+2}+\frac{5}{s+5}\right] =
\left(-4e^{-2t} + 5e^{-5t}\right)\delta_{-1}(t)
$$

In alternativa alla tecnica dei residui si può sfruttare il principio di
identità dei polinomi
$$\begin{aligned}
F(s) &= \frac{s-10}{(s+2)(s+5)} = \frac{R_1}{s+2} + \frac{R_2}{s+5} =
\frac{R_1(s+5)+R_2(s+2)}{(s+2)(s+5)} =\\
&=\frac{(R_1+R_2)s+(5R_1+2R_2)}{(s+2)(s+5)} \Rightarrow \left\{
\begin{aligned}
R_1+R_2 &=1\\
5R_1+2R_2 &= -10
\end{aligned}\right. = \ldots \\&=
\left\{\begin{aligned}
R_1 &= -4 \\ R_2 &= 5
\end{aligned}\right.
\end{aligned}$$

Analogamente è possibile sostituire $n$ valori differenti di $s$ e risolvere il
sistema ottenuto per ricavare i residui.

\newpage
\subsubsection{Radici coincidenti}
La seguente funzione ha due poli
$$
\left[\begin{aligned}
p_1 &= 0\\
p_2 &= -3 \qquad m_a=2\\
n-m &=2
\end{aligned}\right.
$$
$$
F(s) = \frac{s+18}{s(s+3)^2} = \frac{R_1}{s} + \frac{R_2^{(1)}}{s+3} +
\frac{R_2^{(2)}}{(s+3)^2}
$$
Si calcolano i residui
$$\begin{aligned}
R_1 &= \lim_{s\to 0} sF(s) = \lim_{s\to 0} \frac{s+18}{(s+3)^2} = \frac{18}{9}
= 2\\
R_2^{(2)} &= \lim_{s\to -3} (s+3)^2 F(s) =
\lim_{s\to -3} \frac{s+18}{s} = \frac{15}{-3} = -5\\
R_2^{(1)} &\stackrel{n-m>1}{=} -\sum_{\begin{aligned}
i&=1\\ i &\neq 2
\end{aligned}}^2 R_i = -R_1 = -2 \qquad \text{oppure}\\
R_2^{(1)} & = \lim_{s\to -3} \frac{d}{ds} \left[(s+3)^2F(s)\right] = -2
\end{aligned}$$
Nella sommatoria vanno calcolati solo i residui semplici e non quelli associati
a potenze superiori all'unità.

$$
f(t) = \Lap^{-1}\left[F(s)\right] = \Lap^{-1} \left[\frac{2}{s} +
\frac{-2}{s+3} + \frac{-5}{(s+3)^2}\right] =
\left(2-2e^{-3t} -5e^{-3t}\cdot t\right)\delta_{-1}(t)
$$

\newpage
\subsection{Radici complesse e coniugate}
$$
F(s) = \frac{100}{(s+1)(s^2 + 4s + 13)}
$$
I poli saranno $p_1 = -1,\ p_2 = -2+j3,\ p_3 = p_2^* = -2-j3 $\\
La decomposizione in fratti semplici
$$
F(s) = \frac{R_1}{s+1} + \frac{R_2}{s+2-j3} + \frac{R_3=R_2^*}{s+2+j3}
$$
$$\begin{aligned}
R_1 &= \lim_{s\to -1} (s+1)F(s) = \lim_{s\to -1}\frac{100}{s^2+4s +13} =
\frac{100}{10} = 10\\
R_2 &= \lim_{s\to -2+j3} (s+2-j3)F(s) =\lim_{s\to -2+j3}
\frac{100}{(s+1)(s+2+j3)} = \\
&=\frac{100}{(-2 + j3 + 1)(\cancel{-2}+j3 \cancel{+
2} + j3)} = -\frac{5}{3}(3-j) = -5 + \frac{5}{3}j=R_3^*\\
R_3 &= -5 - \frac{5}{3}j
\end{aligned}$$
Si può dunque antitrasformare
$$\begin{aligned}
f(t) &= \Lap^{-1} [f(s)] = \Lap^{-1}
\left[
\frac{10}{s+1} - \frac{5}{3}\left(
\frac{3-j}{s+2-j3}+ \frac{3+j}{s+2+j3}
\right)
\right] = \\
&=10\left(e^{-t} -\frac{1}{6}\cdot
2|R_2|e^{\Re(p_2)t}\cos\left(\Im(p_2)\cdot t + \text{arg}(p_2)\right)
\right) = \\
&= 10\left(e^{-t} -\frac{1}{6}2\sqrt{10}e^{-2t} \cos(3t+
\text{arg}(-3+j))\right)\delta_{-1}(t)
\end{aligned}$$

In alternativa sfruttando il principio di identità dei polinomi
$$
F(s) = \frac{100}{(s+1)(s^2 + 4s + 13)} = \frac{R_1}{s+1} +
\frac{R_as+R_b}{s^2+4s+13}
$$
Si calcola $R_1=10$ con la formula precedente, si esegue il minimo comune
multiplo
$$
100 = 10(s^2+4s + 13)+(s+1)(R_as+R_b)
$$
In questo caso può essere conveniente sostituire due valori di $s$
$$
\begin{aligned}
s=0 &\rightarrow\\
s=1 &\rightarrow
\end{aligned}
\left\{
\begin{aligned}
 100 &= 130 + R_b\\
 100 &= 180 + 2(R_a + R_b)
\end{aligned}\right.
\rightarrow
\left\{
\begin{aligned}
R_a &= -10\\
R_b &= -30
\end{aligned}
\right.
$$

\newpage
Si può calcolare l'antitrasformata
$$
f(t) = \Lap^{-1} [F(s)] = \Lap^{-1} \left[
\frac{10}{s+1} -10\frac{s+3}{(s+2)^2+9}
\right]
$$
Il secondo termine può essere antitrasformato facendo comparire il $3$ ed
$(s+2)$ in modo da far comparire le trasformate notevoli di seno e coseno.
$$
\frac{s+3  }{(s+2)^2 + 9}  =\frac{(s+2)+1}{(s+2)^2 + 3^2}
$$
$$\begin{aligned}
f(t) &=
\Lap^{-1}\left[\frac{10}{s+1}-10\frac{s+2}{(s+2)^2+3^2}-\frac{10}{3}\frac{3
} {(s+2)^2 +3^2} \right] =\\
&= 10\left(e^{-t} -e^{-2t} \left( \cos(3t)+\frac{1}{3}\sin(3t) \right)
\right)\delta_{-1}(t)
\end{aligned}$$
Con varie formule trigonometriche è possibile ottenere l'espressione identica
alla precedente $\sqrt{10}\cos(3t+
\text{arg}(-3+j))$.
33:22
